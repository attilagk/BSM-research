\documentclass[letterpaper]{article}
%\documentclass[12pt,letterpaper]{article}
%\setlength{\textwidth}{480pt}
%\setlength{\textheight}{630pt}
%\setlength{\voffset}{0pt}

\usepackage{amsmath, geometry, graphicx}
\usepackage{natbib}
%\usepackage{float}
\bibliographystyle{plainnat}

% https://tex.stackexchange.com/questions/6758/how-can-i-create-a-bibliography-like-a-section
%\usepackage{etoolbox}
%\patchcmd{\thebibliography}{\section*}{\section}{}{}

\pagestyle{plain}

\title{Involvement of Brain Somatic Mosaicism in Schizophrenia}

\author{Attila Jones\(^{1,2}\), Chaggai Rosenbluh\(^{1}\), ..., Andrew Chess\(^{1,2,3,4,\ast}\)}

\date{Icahn School of Medicine at Mount Sinai (ISMMS)}

\begin{document}

\maketitle

\begin{description}
\item[1] Department of Cell, Developmental and Regenerative Biology, ISMMS 
\item[2] Institute for Genomics and Multiscale Biology, Department of Genetics and Genomic Sciences, ISMMS 
\item[3] Neuroscience Program, The Graduate School of Biomedical Sciences, ISMMS 
\item[4] Division of Psychiatric Genomics, Department of Psychiatry, ISMMS 
%\item[\(\S\)] full list of consortium members appears in the Author
Information section
\item[\(\ast\)] correspondence: andrew.chess@mssm.edu 
\end{description}

\clearpage

\section*{Abstract}

The abstract

\section*{Introduction}

Schizophrenia is a psychiatric disorder, which... heritability: 0.8 ...\citep{Kahn2015}.  
GWAS (reviewed by \cite{Visscher2017}) has been essential in clarifying the
genetic of architecture of schizophrenia~\citep{Ripke2014,Pardinas2018} and other psychiatric disorders
\citep{Sullivan2012}: they are polygenic and much of the heritability is due to
a large set of common germline variants, typically SNPs, with
small effect size.  GWAS also revealed that psychiatric disorders
typically share much heritability with each
other~\citep{Consortium2009,PsychiatricGenomicsConsortium2013} and to a lesser extent with
some neurological disorders~\citep{Consortium2018} indicating shared
pathological mechanisms and supporting the neurodevelopmental hypothesis of
schizophrenia~\citep{Nour2015}.

Beyond common SNPs other classes of genetic variation have also been shown to
carry substantial risk for schizophrenia: rare CNVs~\citep{Rees2014}, rare germline
SNVs~\citep{Purcell2014,Singh2017} as well as de novo
SNVs~\citep{Fromer2014,Rees2020}.  These studies confirmed the polygenicity of
schizophrenia and found excess mutational burden in several functional
gene sets such as synaptic and voltage gated calcium channel and FMRP target
genes\citep{Fromer2014,Purcell2014} ... and loss of function intolerant
genes~\citep{Rees2020,Singh2017} but found no evidence for increased genome
wide germline single nucleotide mutation rate in schizophrenics.

Despite much progress translation of the results of the aforementioned studies
into effective medical diagnostic and treatment methods for schizophrenia is
still lacking~\citep{Breen2016,Foley2017}. Increasing sample sizes, especially
for currently underrepresented ethnic groups, is expected to uncover new
potential risk genes and therapeutic targets~\citep{Visscher2017}, but the difficulty of interpreting
polygenic signals from SNP arrays will persist~\citep{Boyle2017}.  Rare and de
novo variants detected by sequencing afford somewhat easier interpretation.
However, besides germline variants a number of distinct genetic and epigenetic
processes are hypothesized to play a role in schizophrenia and psychiatric
disorders~\citep{PsychENCODEConsortium2015}.

Somatic variants have been called the dark matter of psychiatric
genetics~\citep{Insel2014}.  Their systematic, genome-wide study has just begun
with the announcement of the Brain Somatic Mosaicism
project~\citep{McConnell2017}.  Within that project we and others recently
developed new best practices and tools for calling somatic variants from brain
samples based on bulk tissue sequencing data~\citep{Wang2021}.  In the present
work, for the first time, we apply these analytical tools to deep, whole
genome sequencing data from the DLPFC of \(m\) schizophrenic and \(n\) control
individuals.

\section*{Results}

\subsection*{Somatic SNV callsets}

\subsubsection*{Genome wide somatic SNV call rate for Control and Schizophrenic samples}

[notebook/2020-06-10-annotate-explore-variants]

We extracted NeuN+ nuclei from postmortem samples of the DLPFC of 40 control
and 61 schizophrenic individuals and subjected them to short read DNA
sequencing at 50--540\(\times\) coverage.  Next we called somatic SNVs using a
pipeline developed and validated by the BSMN consortium~\citep{Wang2021}.  The
genome-wide somatic SNV call rate was similar for control and schizophrenic
samples (median around 40) and showed great variability, ranging 0 to 100 with
the exception of an outlier schizophrenic sample reaching \(\approx 200\).


%\begin{figure}
\includegraphics[scale=0.6]{../../notebook/2020-06-10-annotate-explore-variants/named-figure/numcalls-Dx.pdf}

%\caption{Genome wide somatic SNV call rate for Control and Schizophrenic samples.  TODO:
%	add Harvard control samples}
%\end{figure}

\subsubsection*{Genome wide somatic SNV call rate depends on sequencing read depth}

[notebook/2020-12-09-walsh-data]

The finding that somatic SNV call rate is unaltered in schizophrenics was
expected based on similar results for rare germline and de-novo
SNVs~\citep{Purcell2014,Singh2017,Fromer2014,Rees2020}.
As for the high variability of somatic SNV call rate we sought 
biological or technical variables explaining that variation.  As the figure
shows sequencing read depth explains much of
the variability in call rate, which indicates that the variant calling
pipeline adapts to lower coverage by applying stronger filters in order to 
maintain high specificity.

%\begin{figure}
\includegraphics[scale=0.6]{../../notebook/2020-12-09-walsh-data/named-figure/ncalls-depth-chess-walsh.pdf}

%\caption{Genome wide somatic SNV call rate depends on sequencing read depth}
%\end{figure}

\subsubsection*{Distribution of allele frequency and read depth}

[notebook/2020-08-13-select-vars]

The joint distribution of read depth and estimated allele frequency across all
4141 somatic SNVs in our data set showed that increasing read depth allowed
calling SNVs at lower frequencies.  (Note that even though allele frequency is
a continuous variable its estimate is discrete, which is an inherent feature
of the pipeline.)  As a result the conditional density of allele frequency is
shifted to the left at higher read depth relative to lower depth.  If we
assume roughly constant somatic point mutation rate along the cell lineages of
our samples~\citep{Rodin2021} then the expected distribution of allele
frequency is quasi exponential.  With decreasing read depth the observed
conditional allele frequency distribution departed increasingly from an
exponential distribution, which indicates that the pipeline's sensitivity for
low allele frequency (below c.a 0.05) somatic SNVs is limited at even high
(\(\approx\) 400) sequencing coverage.

%\begin{figure}
A

\includegraphics[scale=0.6]{../../notebook/2020-08-13-select-vars/named-figure/DP-AF-jointdensity-calls.pdf}

B

\includegraphics[scale=0.6]{../../notebook/2020-08-13-select-vars/named-figure/DP-AF-jointdensity-conddensity.pdf}
%\caption{Distribution of allele frequency and read depth.  TODO: omit upper
%left graph (A) and merge (A) and (B)}
%\end{figure}

\subsection*{Enrichment analysis}


Studies of rare germline and de-novo SNVs showed that,
although the genome-wide ``background'' mutation rate is unaltered in
schizophrenia, specific genomic elements and gene sets bear increased burden
of---in other words: are enriched in---SNVs in schizophrenic individuals relative to
controls.  We sought to test enrichment in several genomic elements or gene
sets (see Table below).  To that end we established statistical procedures.

\subsubsection*{Pairwise joint distributions of response and explanatory variables}

[notebook/2021-01-26-indiv-counts-modeling]

For a genomic element or gene set  \(S\) we inspected the joint distribution,
across all study subjects, of disease status Dx, two enrichment variables
\(n_S\), \(f_S\), and technical and biological variables that potentially
carry information on those enrichment variables.  The enrichment variable
\(n_S\) is the number of somatic SNVs located in \(S\), whereas the \(f_S =
	n_S/n_\mathrm{genome}\) is the fraction of somatic SNVs in \(S\)
relative to all somatic SNVs in the genome.  The aim was to formally test
enrichment in \(S\) in schizophrenia by regressing the response variables
\(n_S\) or \(f_S\) on the explanatory variable Dx
and other, ``confounding'', variables.

In case of the plot matrix below \(S\) is the set of schizophrenia GWAS genes
defined as all genes, including their introns, that are located fully or
partially in schizophrenia GWAS loci (Table S4 in \cite{Pardinas2018}).

Notable observations:
\begin{itemize}
	\item the strongsest dependence was seen between \(n_S\) and \(n_\mathrm{genome}\) as expected
		when the genome-wide background rate of somatic SNVs is
		substantial relative to enrichment in \(S\)
	\item we have already noted the strong dependence between read depth DP and
		\(n_\mathrm{genome}\)
        \item other dependencies were relatively weak
	\item for many control individuals ageOfDeath is below 50 years while for
		most schizophrenic individuals it is above 50.  This is
		because the Mount Sinai dataset contained older individuals
		that were either control or schizophrenic but the Harvard dataset contained
		only younger control individuals
\end{itemize}

%\begin{figure}
\includegraphics[scale=0.6]{../../notebook/2021-01-26-indiv-counts-modeling/named-figure/symmetric-scatter-matrix_scz.pdf}

%\caption{Pairwise joint distributions of response and explanatory variables
%	TODO: fix legend}
%\end{figure}

\subsubsection*{Modeling the number (A) and fraction (B) of somatic SNV calls in a gene set}

[notebook/2021-01-26-indiv-counts-modeling]

We considered two generalized linear model families depending on the response
variable:
\begin{description}
\item[response is \(f_S\):] logistic family: binomial error distribution with
	logistic link function; see panel (A)
\item[response is \(n_S\):] log-linear family: Poisson error distribution with exponential link function; see panel (B)
\end{description}

Within each family we considered relatively simple or complex models depending
on the set of variables selected as explanatory.  For \(S =\) schizophrenia
GWAS genes, and within the
logistic family, panel (A) shows fitted lines under both \(f_S \sim 1 +
	\mathrm{Dx}\) and \(f_S \sim 1 + \mathrm{Dx} + \mathrm{ageOfDeath}\)
(panel (A)), where \(1\) means intercept.  For the log-linear family panel (B)
shows fitted lines under both \(n_S \sim 1 + \mathrm{Dx}\) and \(n_S \sim 1 +
	\mathrm{Dx} + \(n_\mathrm{genome}\)\)

%\begin{figure}
A\includegraphics[scale=0.45]{../../notebook/2021-01-26-indiv-counts-modeling/named-figure/scz_gwas_genes-ageOfDeath-binom-fit_scz.pdf}
B\includegraphics[scale=0.45]{../../notebook/2021-01-26-indiv-counts-modeling/named-figure/scz_gwas_genes-log10_ncalls-poisson-fit_scz.pdf}

%\caption{Modeling the number (A) and fraction (B) of somatic SNV calls in a
%gene set}
%\end{figure}

\subsubsection*{Variable selection for regression models}

[notebook/2021-01-26-indiv-counts-modeling]

To select the best fitting model within each family we applied forward
variable selection algorithm with the Akaike information criterion (AIC) as
optimality criterion.  (The Bayesian (BIC) information instead of AIC gave the
same results.)

Still in the case of \(S =\) schizophrenia GWAS genes we see that the first,
thus most informative, variable selected is Dx under the logistic model (panel A).
Any additional variables increased the information criteria and thus lead to
overfitting the data.

Panel (B) shows the same result under the log-linear model except that the
most informative variable is \(n_\mathrm{genome}\), which was expected by its
strong correlation with \(n_S\).  Nonetheless, Dx is selected also under this
model family.

%\begin{figure}
A

\includegraphics[scale=0.5]{../../notebook/2021-01-26-indiv-counts-modeling/named-figure/scz_gwas_genes-fw-varsel-binom-onlyIC_scz.pdf}

B

\includegraphics[scale=0.5]{../../notebook/2021-01-26-indiv-counts-modeling/named-figure/scz_gwas_genes-fw-varsel-pois-onlyIC_scz.pdf}
%\caption{Variable selection for regression models. A) Binomial--logistic
%	family.  B) Poisson--log-linear family.}
%\end{figure}

\subsubsection*{The effect of ageOfDeath and that of Dataset are mediated by disease status Dx}

[notebook/2021-01-26-indiv-counts-modeling]

As we noted one of our two datasets contained relatively young individuals who
were all control.  This leads to the causal-probabilistic graphical model that
\(\mathrm{ageOfDeath} \leftarrow \mathrm{Dataset} \rightarrow \mathrm{Dx}
	\rightarrow y\)\), where \(y\) is the response enrichment variable
\(f_S\) or \(n_S\).  It means that ageOfDeath and
Dataset have indirect effects on enrichment such that these effects are
mediated by disease status Dx. If this model is true then \(Dx\) d-separates
\(y\) from Dataset and ageOfDeath so that \(y\) becomes independent from
ageOfDeath and Dataset if we condition on Dx = control or Dx =
schizophrenia.  Panel (A) illustrates that.  The important consequence is
that we should remove ageOfDeath and Dataset from the set from our models as long
as we include Dx.  Failing to do so may confound our
inference on the effect of Dx on enrichment (Panel B).

%\begin{figure}
A\includegraphics[scale=0.45]{../../notebook/2021-01-26-indiv-counts-modeling/named-figure/Dx_Dataset_ageOfDeath_scz.pdf}
B\includegraphics[scale=0.45]{../../notebook/2021-01-26-indiv-counts-modeling/named-figure/collinearity_Dx_Dataset_ageOfDeath_scz.pdf}
%\caption{The effect of ageOfDeath and that of Dataset are mediated by disease status Dx.  H: Harvard Dataset,
%S: Mount Sinai Dataset}
%\end{figure}

\subsubsection*{Residual analysis to assess model fit }

[notebook/2021-01-26-indiv-counts-modeling]

%\begin{figure}
\includegraphics[width=0.9\textwidth]{../../notebook/2021-01-26-indiv-counts-modeling/named-figure/resid-binom.pdf}
%\caption{Residual analysis to assess model fit for genomic elements and gene
%sets of increasing size (from top left to bottom right)}
%\end{figure}

\subsubsection*{Genomic elements and gene sets}

[notebook/2021-01-26-indiv-counts-modeling]

%\begin{table}
\begin{tabular}{lrlr}
category & genomic element, gene set & selected model formula & size \\
\hline
\hline
damaging & coding nonsyn & 1 & 55 \\
& coding syn & 1 + Gender & 38 \\
& missense & None & 0 \\
& stop gain & None & 1 \\
& splice site & None & 1 \\
& coding & 1 & 80 \\
& deleterious (sift) & 1 & 19 \\
& damaging (polyphen) & 1 + DP & 9 \\
\hline
regulartory & TFBS & 1 + AF & 93 \\
& CpG island & 1 + AF + BaseQRankSum & 57 \\
& enhancer & 1 & 99 \\
& genic enhancer & 1 & 22 \\
\hline
constraint & evolutionary cons. (phast) & 1 + DP + AF + Gender & 117 \\
& evolutionary cons. (gerp) & 1 + Dx & 127 \\
& LoF intolerant genes & 1 + Dx & 143 \\
\hline
SCZ gene sets & PSD & 1 & 95 \\
& FMRP-target & 1 + AF & 118 \\
& SCZ de novo & 1 + AF & 97 \\
& Calcium channel & 1 + Dx & 13 \\
& composite & 1 & 126 \\
\hline
GWAS gene sets & ADHD GWAS & 1 + AF + DP & 6 \\
& Anorexia nervosa GWAS & None & 4 \\
& ASD GWAS & 1 + Dx & 27 \\
& Bipolar disorder GWAS & 1 + Dx & 77 \\
& MDD GWAS & 1 & 48 \\
& Schizophrenia GWAS & 1 + Dx & 72 \\
& Psychiatric GWAS & 1 + Dx & 105 \\
\end{tabular}
%\caption{Genomic elements and gene sets.  Size is defined as the number of
%	individuals with at least one somatic SNV call in the genomic element
%	or gene set. TODO: remove ``coding'' since it's redundant give ``coding
%	nonsyn'' and ``coding syn''}
%\end{table}

\subsubsection*{Enrichment of genomic elements and gene sets in somatic SNVs in schizophrenia}

[notebook/2021-01-26-indiv-counts-modeling]

%\begin{figure}
\includegraphics[width=0.9\textwidth]{../../notebook/2021-01-26-indiv-counts-modeling/named-figure/enrichment-5-plots_scz.pdf}
%\caption{Enrichment of genomic elements and gene sets in somatic SNVs in schizophrenia}
%\end{figure}

\section*{Discussion}


\section*{Methods}


\subsection*{Data availability}

Data and analytical results...

\subsection*{Code availability}

All code developed by A.~Jones is available at:\\
https://github.com/attilagk/BSM-research

\bibliography{bsm}
%\begin{thebibliography}{10}
%
%\bibitem{Noor2015}
%Abdul Noor, Lucie Dupuis, Kirti Mittal, Anath~C. Lionel, Christian~R. Marshall,
%  Stephen~W. Scherer, Tracy Stockley, John~B. Vincent, Roberto Mendoza-Londono,
%  and Dimitri~J. Stavropoulos.
%\newblock 15q11.2 duplication encompassing only the {UBE}3a gene is associated
%  with developmental delay and neuropsychiatric phenotypes.
%\newblock 36(7):689--693.
%
%\bibitem{Rees2014}
%Elliott Rees, James T~R Walters, Lyudmila Georgieva, Anthony~R Isles,
%  Kimberly~D Chambert, Alexander~L Richards, Gerwyn Mahoney-Davies, Sophie~E
%  Legge, Jennifer~L Moran, Steven~A McCarroll, Michael~C O'Donovan, Michael~J
%  Owen, and George Kirov.
%\newblock {Analysis of copy number variations at 15 schizophrenia-associated
%  loci.}
%\newblock {\em The British journal of psychiatry : the journal of mental
%  science}, 204(2):108--14, feb 2014.
%
%\bibitem{Gregg2010a}
%Christopher Gregg, Jiangwen Zhang, Brandon Weissbourd, Shujun Luo, Gary~P
%  Schroth, David Haig, and Catherine Dulac.
%\newblock {High-resolution analysis of parent-of-origin allelic expression in
%  the mouse brain.}
%\newblock {\em Science (New York, N.Y.)}, 329(5992):643--8, aug 2010.
%
%\bibitem{Andergassen2017}
%Daniel Andergassen, Christoph~P Dotter, Daniel Wenzel, Verena Sigl, Philipp~C
%  Bammer, Markus Muckenhuber, Daniela Mayer, Tomasz~M Kulinski, Hans-Christian
%  Theussl, Josef~M Penninger, Christoph Bock, Denise~P Barlow, Florian~M
%  Pauler, and Quanah~J Hudson.
%\newblock Mapping the mouse allelome reveals tissue-specific regulation of
%  allelic expression.
%\newblock {\em eLife}, 6, August 2017.
%
%\bibitem{Baran2015}
%Yael Baran, Meena Subramaniam, Anne Biton, Taru Tukiainen, Emily~K. Tsang,
%  Manuel~A. Rivas, Matti Pirinen, Maria Gutierrez-Arcelus, Kevin~S. Smith,
%  Kim~R. Kukurba, Rui Zhang, Celeste Eng, Dara~G. Torgerson, Cydney Urbanek,
%  Jin~Billy Li, Jose~R. Rodriguez-Santana, Esteban~G. Burchard, Max~A. Seibold,
%  Daniel~G. MacArthur, Stephen~B. Montgomery, Noah~A. Zaitlen, and Tuuli
%  Lappalainen.
%\newblock {The landscape of genomic imprinting across diverse adult human
%  tissues}.
%\newblock {\em Genome Research}, 25(7), 2015.
%
%\bibitem{DeVeale2012}
%Brian DeVeale, Derek van~der Kooy, and Tomas Babak.
%\newblock {Critical evaluation of imprinted gene expression by RNA-seq: A new
%  perspective}.
%\newblock {\em PLoS Genetics}, 8(3):e1002600, jan 2012.
%
%\bibitem{Perez2015}
%Julio~D Perez, Nimrod~D Rubinstein, Daniel~E Fernandez, Stephen~W Santoro,
%  Leigh~A Needleman, Olivia Ho-Shing, John~J Choi, Mariela Zirlinger,
%  Shau-Kwaun Chen, Jun~S Liu, and Catherine Dulac.
%\newblock {Quantitative and functional interrogation of parent-of-origin
%  allelic expression biases in the brain.}
%\newblock {\em eLife}, 4:e07860, jan 2015.
%
%\bibitem{Fromer2016a}
%Fromer, Menachem and Roussos, Panos and Sieberts, Solveig K and Johnson,
%Jessica S and Kavanagh, David H and Perumal, Thanneer M and Ruderfer, Douglas
%M and Oh, Edwin C and Topol, Aaron and Shah, Hardik R and others.
%\newblock {Gene expression elucidates functional impact of polygenic risk for
%  schizophrenia.}
%\newblock {\em Nature Neuroscience}, 19(11):1442, Sep 2016.
%
%\bibitem{Babak2015}
%Tomas Babak, Brian DeVeale, Emily~K Tsang, Yiqi Zhou, Xin Li, Kevin~S Smith,
%  Kim~R Kukurba, Rui Zhang, Jin~Billy Li, Derek van~der Kooy, Stephen~B
%  Montgomery, and Hunter~B Fraser.
%\newblock Genetic conflict reflected in tissue-specific maps of genomic
%  imprinting in human and mouse.
%\newblock {\em Nature Genetics}, 47:544--549, May 2015.
%
%\bibitem{Hoffman2016}
%Gabriel~E Hoffman and Eric~E Schadt.
%\newblock {variancePartition: interpreting drivers of variation in complex gene
%  expression studies.}
%\newblock {\em BMC Bioinformatics}, 17(1):483, nov 2016.
%
%\bibitem{Gregg2010}
%Christopher Gregg, Jiangwen Zhang, James~E. Butler, David Haig, and Catherine
%  Dulac.
%\newblock {Sex-Specific Parent-of-Origin Allelic Expression in the Mouse
%  Brain}.
%\newblock {\em Science}, 329(5992):682--685, aug 2010.
%
%\bibitem{Crespi2008}
%Bernard Crespi.
%\newblock {Genomic imprinting in the development and evolution of psychotic
%  spectrum conditions.}
%\newblock {\em Biological reviews of the Cambridge Philosophical Society},
%  83(4):441--93, nov 2008.
%
%\bibitem{Sullivan2012}
%Patrick~F Sullivan, Mark~J Daly, and Michael O'Donovan.
%\newblock {Genetic architectures of psychiatric disorders: the emerging picture
%  and its implications}.
%\newblock {\em Nature Reviews Genetics}, 13(8):537--551, aug 2012.
%
%\bibitem{Horvath2013}
%Steve Horvath.
%\newblock {DNA methylation age of human tissues and cell types.}
%\newblock {\em Genome Biology}, 14(10):R115, oct 2013.
%
%\bibitem{Ubeda2012}
%Francisco Ubeda and Andy Gardner.
%\newblock {A model for genomic imprinting in the social brain: elders.}
%\newblock {\em Evolution; international journal of organic evolution},
%  66(5):1567--81, may 2012.
%
%\bibitem{Isles2006}
%Anthony~R Isles, William Davies, and Lawrence~S Wilkinson.
%\newblock {Genomic imprinting and the social brain.}
%\newblock {\em Philosophical transactions of the Royal Society of London.
%  Series B, Biological sciences}, 361(1476):2229--37, dec 2006.
%
%\end{thebibliography}


\section*{Acknowledgements}

We thank...

\section*{Author Information}

\section*{Author Contributions}

A.J...

\section*{Competing Interests}

None

\section*{Figure Legends}

\subsection*{Figure 1}
%\includegraphics[width=1.0\textwidth]{figures/by-me/commonmind-rna-seq-ms/commonmind-rna-seq-ms.pdf}

\section*{Tables}

\end{document}
