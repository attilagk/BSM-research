% You can submit either a single PDF file that includes the manuscript text
% and any display items, or separate files for text, figures and tables.
%
% 1500 words, excluding the introductory paragraph, online Methods,
% references and figure legends
%
% TODOs
% - cover letter

\documentclass[letterpaper]{article}
%\documentclass[12pt,letterpaper]{article}
%\setlength{\textwidth}{480pt}
%\setlength{\textheight}{630pt}
%\setlength{\voffset}{0pt}

\usepackage{amsmath, geometry, graphicx}
\usepackage{natbib}
%\usepackage{float}
\bibliographystyle{plainnat}

% https://tex.stackexchange.com/questions/6758/how-can-i-create-a-bibliography-like-a-section
%\usepackage{etoolbox}
%\patchcmd{\thebibliography}{\section*}{\section}{}{}

\pagestyle{plain}

\title{Involvement of Brain Somatic Mosaicism in Schizophrenia}

\author{Attila Jones\(^{1,2,\ddagger}\), Chaggai Rosenbluh\(^{1}\), ..., Andrew Chess\(^{1,2,3,4,\ast}\)}

\date{Icahn School of Medicine at Mount Sinai (ISMMS)}

\begin{document}

\maketitle

\begin{description}
\item[1] Department of Cell, Developmental and Regenerative Biology, ISMMS 
\item[2] Institute for Genomics and Multiscale Biology, Department of Genetics and Genomic Sciences, ISMMS 
\item[3] Neuroscience Program, The Graduate School of Biomedical Sciences, ISMMS 
\item[4] Division of Psychiatric Genomics, Department of Psychiatry, ISMMS 
%\item[\(\S\)] full list of consortium members appears in the Author
Information section
\item[\(\ast\)] correspondence: andrew.chess@mssm.edu 
\end{description}

\clearpage

\section*{Abstract}

The abstract

\section*{Introduction}

Schizophrenia is a psychiatric disorder, which... heritability: 0.8 ...\cite{Kahn2015}.  
GWAS (reviewed by \cite{Visscher2017}) has been essential in clarifying the
genetic of architecture of schizophrenia~\cite{Ripke2014,Pardinas2018} and other psychiatric disorders
\cite{Sullivan2012}: they are polygenic and much of the heritability is due to
a large set of common germline variants, typically SNPs, with
small effect size.  GWAS also revealed that psychiatric disorders
typically share much heritability with each
other~\cite{Consortium2009,PsychiatricGenomicsConsortium2013} and to a lesser extent with
some neurological disorders~\cite{Consortium2018} indicating shared
pathological mechanisms and supporting the neurodevelopmental hypothesis of
schizophrenia~\cite{Nour2015}.

Beyond common SNPs other classes of genetic variation have also been shown to
carry substantial risk for schizophrenia: rare CNVs~\cite{Rees2014}, rare germline
SNVs~\cite{Purcell2014,Singh2017} as well as de novo
SNVs~\cite{Fromer2014,Rees2020}.  These studies confirmed the polygenicity of
schizophrenia and found excess mutational burden in several functional
genesets such as synaptic and voltage gated calcium channel and FMRP target
genes\cite{Fromer2014,Purcell2014} ... and loss of function intolerant
genes~\cite{Rees2020,Singh2017} but found no evidence for increased genome
wide germline single nucleotide mutation rate in schizophrenics.

Despite much progress translation of the results of the aforementioned studies
into effective medical diagnostic and treatment methods for schizophrenia is
still lacking~\cite{Breen2016,Foley2017}. Increasing sample sizes, especially
for currently underrepresented ethnic groups, is expected to uncover new
potential risk genes and therapeutic targets~\cite{Visscher2017}, but the difficulty of interpreting
polygenic signals from SNP arrays will persist~\cite{Boyle2017}.  Rare and de
novo variants detected by sequencing afford somewhat easier interpretation.
However, besides germline variants a number of distinct genetic and epigenetic
processes are hypothesized to play a role in schizophrenia and psychiatric
disorders~\cite{PsychENCODEConsortium2015}.

Somatic variants have been called the dark matter of psychiatric
genetics~\cite{Insel2014}.  Their systematic, genome-wide study has just begun
with the announcement of the Brain Somatic Mosaicism
project~\cite{McConnell2017}.  Within that project we and others recently
developed new best practices and tools for calling somatic variants from brain
samples based on bulk tissue sequencing data~\cite{Wang2021}.  In the present
work, for the first time, we apply these analytical tools to deep, whole
genome sequencing data from the DLPFC of \(m\) schizophrenic and \(n\) control
individuals.

%A related question is the number of imprinted genes in the human brain.  Some
%1,300 genes were estimated to be imprinted in the mouse
%brain~\cite{Gregg2010a} but followup studies using mouse or human subjects
%arrived at estimates that are lower with an order of
%magnitude~\cite{Andergassen2017,Babak2015,Baran2015,DeVeale2012,Perez2015}.

\section*{Results}

\subsection*{Identification of imprinted genes in the adult human brain}

\section*{Discussion}


\section*{Methods}


\subsection*{Data availability}

Data and analytical results...

\subsection*{Code availability}

All code developed by A.~Jones is available at:\\
https://github.com/attilagk/BSM-research

\bibliography{bsm}
%\begin{thebibliography}{10}
%
%\bibitem{Noor2015}
%Abdul Noor, Lucie Dupuis, Kirti Mittal, Anath~C. Lionel, Christian~R. Marshall,
%  Stephen~W. Scherer, Tracy Stockley, John~B. Vincent, Roberto Mendoza-Londono,
%  and Dimitri~J. Stavropoulos.
%\newblock 15q11.2 duplication encompassing only the {UBE}3a gene is associated
%  with developmental delay and neuropsychiatric phenotypes.
%\newblock 36(7):689--693.
%
%\bibitem{Rees2014}
%Elliott Rees, James T~R Walters, Lyudmila Georgieva, Anthony~R Isles,
%  Kimberly~D Chambert, Alexander~L Richards, Gerwyn Mahoney-Davies, Sophie~E
%  Legge, Jennifer~L Moran, Steven~A McCarroll, Michael~C O'Donovan, Michael~J
%  Owen, and George Kirov.
%\newblock {Analysis of copy number variations at 15 schizophrenia-associated
%  loci.}
%\newblock {\em The British journal of psychiatry : the journal of mental
%  science}, 204(2):108--14, feb 2014.
%
%\bibitem{Gregg2010a}
%Christopher Gregg, Jiangwen Zhang, Brandon Weissbourd, Shujun Luo, Gary~P
%  Schroth, David Haig, and Catherine Dulac.
%\newblock {High-resolution analysis of parent-of-origin allelic expression in
%  the mouse brain.}
%\newblock {\em Science (New York, N.Y.)}, 329(5992):643--8, aug 2010.
%
%\bibitem{Andergassen2017}
%Daniel Andergassen, Christoph~P Dotter, Daniel Wenzel, Verena Sigl, Philipp~C
%  Bammer, Markus Muckenhuber, Daniela Mayer, Tomasz~M Kulinski, Hans-Christian
%  Theussl, Josef~M Penninger, Christoph Bock, Denise~P Barlow, Florian~M
%  Pauler, and Quanah~J Hudson.
%\newblock Mapping the mouse allelome reveals tissue-specific regulation of
%  allelic expression.
%\newblock {\em eLife}, 6, August 2017.
%
%\bibitem{Baran2015}
%Yael Baran, Meena Subramaniam, Anne Biton, Taru Tukiainen, Emily~K. Tsang,
%  Manuel~A. Rivas, Matti Pirinen, Maria Gutierrez-Arcelus, Kevin~S. Smith,
%  Kim~R. Kukurba, Rui Zhang, Celeste Eng, Dara~G. Torgerson, Cydney Urbanek,
%  Jin~Billy Li, Jose~R. Rodriguez-Santana, Esteban~G. Burchard, Max~A. Seibold,
%  Daniel~G. MacArthur, Stephen~B. Montgomery, Noah~A. Zaitlen, and Tuuli
%  Lappalainen.
%\newblock {The landscape of genomic imprinting across diverse adult human
%  tissues}.
%\newblock {\em Genome Research}, 25(7), 2015.
%
%\bibitem{DeVeale2012}
%Brian DeVeale, Derek van~der Kooy, and Tomas Babak.
%\newblock {Critical evaluation of imprinted gene expression by RNA-seq: A new
%  perspective}.
%\newblock {\em PLoS Genetics}, 8(3):e1002600, jan 2012.
%
%\bibitem{Perez2015}
%Julio~D Perez, Nimrod~D Rubinstein, Daniel~E Fernandez, Stephen~W Santoro,
%  Leigh~A Needleman, Olivia Ho-Shing, John~J Choi, Mariela Zirlinger,
%  Shau-Kwaun Chen, Jun~S Liu, and Catherine Dulac.
%\newblock {Quantitative and functional interrogation of parent-of-origin
%  allelic expression biases in the brain.}
%\newblock {\em eLife}, 4:e07860, jan 2015.
%
%\bibitem{Fromer2016a}
%Fromer, Menachem and Roussos, Panos and Sieberts, Solveig K and Johnson,
%Jessica S and Kavanagh, David H and Perumal, Thanneer M and Ruderfer, Douglas
%M and Oh, Edwin C and Topol, Aaron and Shah, Hardik R and others.
%\newblock {Gene expression elucidates functional impact of polygenic risk for
%  schizophrenia.}
%\newblock {\em Nature Neuroscience}, 19(11):1442, Sep 2016.
%
%\bibitem{Babak2015}
%Tomas Babak, Brian DeVeale, Emily~K Tsang, Yiqi Zhou, Xin Li, Kevin~S Smith,
%  Kim~R Kukurba, Rui Zhang, Jin~Billy Li, Derek van~der Kooy, Stephen~B
%  Montgomery, and Hunter~B Fraser.
%\newblock Genetic conflict reflected in tissue-specific maps of genomic
%  imprinting in human and mouse.
%\newblock {\em Nature Genetics}, 47:544--549, May 2015.
%
%\bibitem{Hoffman2016}
%Gabriel~E Hoffman and Eric~E Schadt.
%\newblock {variancePartition: interpreting drivers of variation in complex gene
%  expression studies.}
%\newblock {\em BMC Bioinformatics}, 17(1):483, nov 2016.
%
%\bibitem{Gregg2010}
%Christopher Gregg, Jiangwen Zhang, James~E. Butler, David Haig, and Catherine
%  Dulac.
%\newblock {Sex-Specific Parent-of-Origin Allelic Expression in the Mouse
%  Brain}.
%\newblock {\em Science}, 329(5992):682--685, aug 2010.
%
%\bibitem{Crespi2008}
%Bernard Crespi.
%\newblock {Genomic imprinting in the development and evolution of psychotic
%  spectrum conditions.}
%\newblock {\em Biological reviews of the Cambridge Philosophical Society},
%  83(4):441--93, nov 2008.
%
%\bibitem{Sullivan2012}
%Patrick~F Sullivan, Mark~J Daly, and Michael O'Donovan.
%\newblock {Genetic architectures of psychiatric disorders: the emerging picture
%  and its implications}.
%\newblock {\em Nature Reviews Genetics}, 13(8):537--551, aug 2012.
%
%\bibitem{Horvath2013}
%Steve Horvath.
%\newblock {DNA methylation age of human tissues and cell types.}
%\newblock {\em Genome Biology}, 14(10):R115, oct 2013.
%
%\bibitem{Ubeda2012}
%Francisco Ubeda and Andy Gardner.
%\newblock {A model for genomic imprinting in the social brain: elders.}
%\newblock {\em Evolution; international journal of organic evolution},
%  66(5):1567--81, may 2012.
%
%\bibitem{Isles2006}
%Anthony~R Isles, William Davies, and Lawrence~S Wilkinson.
%\newblock {Genomic imprinting and the social brain.}
%\newblock {\em Philosophical transactions of the Royal Society of London.
%  Series B, Biological sciences}, 361(1476):2229--37, dec 2006.
%
%\end{thebibliography}

\section*{Acknowledgements}

We thank...

\section*{Author Information}

\section*{Author Contributions}

A.J...

\section*{Competing Interests}

None

\section*{Figure Legends}

\subsection*{Figure 1}
%\includegraphics[width=1.0\textwidth]{figures/by-me/commonmind-rna-seq-ms/commonmind-rna-seq-ms.pdf}

\section*{Tables}

\end{document}
