\documentclass[letterpaper]{article}
%\documentclass[12pt,letterpaper]{article}
%\setlength{\textwidth}{480pt}
%\setlength{\textheight}{630pt}
%\setlength{\voffset}{0pt}

\usepackage{amsmath, geometry, graphicx}
\usepackage{natbib}
%\usepackage{float}
\bibliographystyle{plainnat}

% https://tex.stackexchange.com/questions/6758/how-can-i-create-a-bibliography-like-a-section
%\usepackage{etoolbox}
%\patchcmd{\thebibliography}{\section*}{\section}{}{}

\pagestyle{plain}

\title{Involvement of Brain Somatic Mosaicism in Schizophrenia}

\author{Attila Jones\(^{1,2}\), Eduardo Maury\(^{3}\), Chaggai Rosenbluh\(^{1,4}\), ..., Andrew Chess\(^{1,2,5,6,\ast}\)}

\date{Icahn School of Medicine at Mount Sinai (ISMMS)}

\begin{document}

\maketitle

\begin{description}
\item[1] Department of Cell, Developmental and Regenerative Biology, ISMMS 
\item[2] Institute for Genomics and Multiscale Biology, Department of Genetics and Genomic Sciences, ISMMS 
\item[3] Harvard Medical School
\item[4] Department of Genetics, Hadassah Medical Center, Jerusalem, Israel
\item[5] Neuroscience Program, The Graduate School of Biomedical Sciences, ISMMS 
\item[6] Division of Psychiatric Genomics, Department of Psychiatry, ISMMS 
%\item[\(\S\)] full list of consortium members appears in the Author
Information section
\item[\(\ast\)] correspondence: andrew.chess@mssm.edu 
\end{description}

\clearpage

\section*{Abstract}

The abstract

\section*{Introduction}

Schizophrenia is a psychiatric disorder, which... heritability: 0.8
...\citep{Kahn2015}.  GWAS (reviewed by \cite{Visscher2017}) has been
essential in clarifying the genetic of architecture of
schizophrenia~\citep{Ripke2014,Pardinas2018} and other psychiatric disorders
\citep{Sullivan2012}: they are polygenic and much of the heritability is due
to a large set of common germline variants, typically SNPs, with small effect
size.  GWASs of schizophrenia have so far yielded 145 genome-wide significant
loci distributed roughly uniformly along the genome involving about 700 genes,
c.a 200 of which are located at the most significant locus in the MHC region,
suggesting that an impaired immune function is responsible for the increased
synaptic pruning seen in schizophrenic brains.  GWAS also revealed that
psychiatric disorders typically share much heritability with each
other~\citep{Consortium2009,PsychiatricGenomicsConsortium2013} and to a lesser
extent with some neurological disorders~\citep{Consortium2018} indicating
shared pathological mechanisms and supporting the neurodevelopmental
hypothesis of schizophrenia~\citep{Nour2015}.

Beyond common SNPs other classes of genetic variation have also been shown to
carry substantial risk for schizophrenia: rare CNVs~\citep{Rees2014}, rare germline
SNVs~\citep{Purcell2014,Singh2017} as well as de novo
SNVs~\citep{Fromer2014,Rees2020}.  These studies confirmed the polygenicity of
schizophrenia and found excess mutational burden in several functional
gene sets such as synaptic and voltage gated calcium channel and FMRP target
genes\citep{Fromer2014,Purcell2014} ... and loss of function intolerant
genes~\citep{Rees2020,Singh2017} but found no evidence for increased genome
wide germline single nucleotide mutation rate in schizophrenics.

Despite much progress translation of the results of the aforementioned studies
into effective medical diagnostic and treatment methods for schizophrenia is
still lacking~\citep{Breen2016,Foley2017}. Increasing sample sizes, especially
for currently underrepresented ethnic groups, is expected to uncover new
potential risk genes and therapeutic targets~\citep{Visscher2017}, but the difficulty of interpreting
polygenic signals from SNP arrays will persist~\citep{Boyle2017}.  Rare and de
novo variants detected by sequencing afford somewhat easier interpretation.
However, besides germline variants a number of distinct genetic and epigenetic
processes are hypothesized to play a role in schizophrenia and psychiatric
disorders~\citep{PsychENCODEConsortium2015}.

Somatic variants have been called the dark matter of psychiatric
genetics~\citep{Insel2014}.  Their systematic, genome-wide study has just begun
with the announcement of the Brain Somatic Mosaicism
project~\citep{McConnell2017}.  Within that project we and others recently
developed new best practices and tools for calling somatic variants from brain
samples based on bulk tissue sequencing data~\citep{Wang2021}.  In the present
work, for the first time, we apply these analytical tools to deep, whole
genome sequencing data from the DLPFC of \(m\) schizophrenic and \(n\) control
individuals.

\section*{Results}

\subsection*{Somatic SNV callsets}

We extracted NeuN+ nuclei from postmortem samples of the DLPFC of 40 control
and 61 schizophrenic individuals and subjected them to short read DNA
sequencing at 50--540\(\times\) coverage.  Next we called somatic SNVs using a
pipeline developed and validated by the BSMN consortium~\citep{Wang2021}.  The
genome-wide somatic SNV call rate was similar for control and schizophrenic
samples (median around 40) and showed great variability, ranging 0 to 100
(Fig~\ref{fig:genomewide-calls}A) with the exception of an outlier
schizophrenic sample reaching \(\approx 189\).  (TODO: investigate the outlier
CMC_MSSM_224 further.)

\begin{figure}
\includegraphics[scale=0.6]{../../notebook/2020-12-09-walsh-data/named-figure/genome-wide-callrate.pdf}
\caption{Genome wide somatic SNV call rate for Control and Schizophrenic samples.  TODO:
	add Harvard control samples, remove ASD}
\label{fig:genomewide-calls}
\end{figure}

We sought biological or technical variables that explain that high variance of
genome-wide.  As the bottom panel of Fig~\ref{fig:genomewide-calls} shows,
sequencing read depth explains much of the variability in call rate, which
indicates that the variant calling pipeline adapts to lower coverage by
applying stronger filters in order to maintain high specificity.

Our pipeline provided us with an estimate of allele frequency for each somatic
SNV.  This enabled us to qualitatively assess the simple model that somatic
point mutation rate roughly constant along the cell lineages of our
samples~\citep{Rodin2021} since under this model the density of allele
frequency is quasi exponential.  We found this to be the case at high
(\(\approx 400\) read depth (blue allele frequency density in Fig~\ref{fig:AF}
right panel).  With decreasing read depth, however, the density of allele
frequency is increasingly shifted towards higher frequencies (orange and green
densities, Fig~\ref{fig:AF}) right).  These results indicate that the somatic
SNVs we called follow the expected allele frequency distribution but our
pipeline's sensitivity for low allele frequency (below c.a 0.05) is low at
low-medium sequencing coverage.

\begin{figure}
\includegraphics[scale=0.6]{../../notebook/2020-08-13-select-vars/named-figure/calls-DP-AF-jointdensity-conddensity.pdf}
\caption{
Distribution of allele frequency and read depth.  Note that even though allele frequency is
a continuous variable its estimate is discrete, which is an inherent feature
of the pipeline.
}
\label{fig:AF}
\end{figure}

\subsection*{Increased local rates of somatic SNVs in schizophrenia}

Next we investigated weather the local rate of somatic SNVs falling into
specific functional categories is altered in schizophrenia.  We
tested genic categories such as nonsynonymous and deleterious SNVs;
regulatory categories such as enhancers; constraint
categories such as loss of function intolerant genes; and gene sets affected
by rare and common variants in schizophrenia and related psychiatric
disorders (Table~\ref{tab:modsel}).

For each SNV category we formally tested weather the local rate of somatic SNVs is
affected by disease status.  For that purpose we employed one univariate test
($\chi^2$) and two multivariate regression models---logistic and log-linear
regression, Fig.~S\ref{fig:fitted-lines} ---, which we combined with a forward
variable selection algorithm (Fig.~S\ref{fig:varsel}) on a set of biological
and technical explanatory variables.  These three techniques
lead to highly concordant inferences (compare p-values between the two
rightmost panels in Fig.~\ref{fig:testresults} and compare
Fig.~S\ref{fig:varsel}A vs.~B) suggesting that confounding variables or false
modeling assumptions did not undermine our analyses.

Three categories of somatic SNVs displayed significantly increased local rate
in schizophrenia samples after multipe test correction: Schizophrenia GWAS,
Psychiatric GWAS (both at false discovery rate, FDR, below $\alpha=0.025$) and
ASD GWAS (FDR below $\alpha=0.05$), see Fig.~\ref{fig:testresults}, 
supporting-data-file1.csv and supporting-data-file2.csv.  For four additional categories the forward algorithm selected the
disease status, Dx, variable (Table~\ref{tab:modsel}) and the local rate was
nonsignificantly increased (Fig.~\ref{fig:testresults}) in schizophrenic
samples; these are: Bipolar disorder GWAS, Calcium channel genes, loss of
function (LoF) intolerant genes and evolutionary conserved regions detected by
GERP++.

For the remaining categories in Table~\ref{tab:modsel} no change in local rate
was detected in schizophrenic samples relative to controls.  The
interpretation of such negative result depended on the the somatic SNVs called
in the given category size.  For larger categories (relative to, say,
Schizophrenia GWAS with 72 SNVs called), such as enhancers (99 SNVs) or
FMRP-target genes (118 SNVs), this means no or small (relative to
Schizophrenia GWAS) change in local somatic SNV rate.  For smaller categories,
such as ADHD GWAS (5 SNVs) or the 17 deleterious SNVs inferred by SIFT the
interpretation is open since there were not enough SNVs called to fit
sufficiently complex models well.  Moreover, for very small categories such as
Anorexia GWAS (2 SNVs), the iterative algorithm of the fit failed to converge.
This is seen in Fig.~S\ref{fig:binomial-QQ},~S\ref{fig:poisson-QQ} from the
progressive improvement in the goodness of fit with increasing number of
somatic SNVs called.  This analysis also showed that for the smaller sized ASD
GWAS (22 SNVs) the fit is already adequate and for the relatively large
Schizophrenia GWAS (86 SNVs) and Psychiatric GWAS (176 SNVs) the fit is
excellent, which gives further support to our model based inference regarding
these categories.

The increased local rate of somatic SNVs in some of the gene set categories
was expected to be a mixed effect of increased rate of damaging variants and
unaltered rate of neutral variants. We attempted at testing this expectation
by narrowing down the gene set categories to nonsynonymous SNVs.  However, our
attempt was undermined by too few nonsynonymous SNVs in even the largest
geneset, the LoF intolerant genes, which contained only 30 nonsynonymous SNVs
out of 1077 SNVs and for which the $\chi^2$ test yielded no significant
change in the local somatic SNV rate ($p = 0.7$).  TODO: supplementary table
with nonsyn SNVs for gene sets.

We performed a complementary test of the above question to evade the limited
number somatic SNV calls.  Instead of narrowing down our Schizophrenia GWAS
category we extended it by including SNVs not only in genic but also
intergenic regions of all 145 schizophrenia GWAS loci.  The local somatic SNV
rate was no more significantly increased in schizophrenic samples relative to
controls (Fig~\ref{fig:scz-gwas-genes-vs-loci}).  Of note is that the most
strongly associated locus to schizophrenia is at the MHC region, which is only
represented in the extended Schizophrenia GWAS loci category but in the
Schizophrenia GWAS gene set due to its structural variability.  To address
this discrepancy we narrowed down again the GWAS loci to genic regions but
this time also including the MHC locus.  We found, again, a significantly
increased local rate of somatic SNVs despite the likely inclusion of some
intergenic regions at the MHC locus (Fig.~\ref{fig:scz-gwas-genes-vs-loci}).

\begin{table}
\begin{tabular}{lrlr}
group & SNV category & selected model formula & SNVs called \\
\hline
\hline
genic & coding nonsyn & 1 & 66\\
 & coding syn & 1 + Gender & 36\\
 & stop gain &  & 1\\
 & splice site &  & 1\\
 & deleterious (sift) & 1 & 17\\
 & damaging (polyphen) & 1 + DP & 8\\
regulatory & TFBS & 1 + AF & 126\\
 & CpG island & 1 + AF + BaseQRankSum & 73\\
 & enhancer & 1 & 130\\
 & genic enhancer & 1 & 12\\
constraint & evolutionary cons. (phast) & 1 + DP + AF + Gender & 276\\
 & evolutionary cons. (gerp) & 1 + Dx & 402\\
 & LoF intolerant genes & 1 + Dx & 1077\\
rare var.~gene sets  & PSD & 1 & 112\\
 & FMRP-target & 1 + AF & 227\\
 & SCZ de novo & 1 + AF & 112\\
 & Calcium channel & 1 + Dx & 11\\
 & composite & 1 & 377\\
GWAS gene set & ADHD GWAS & 1 + AF + DP & 5\\
 & Anorexia nervosa GWAS &  & 2\\
 & ASD GWAS & 1 + Dx & 22\\
 & Bipolar disorder GWAS & 1 + Dx & 91\\
 & MDD GWAS & 1 & 40\\
 & Schizophrenia GWAS & 1 + Dx & 86\\
 & Psychiatric GWAS & 1 + Dx & 176\\
\end{tabular}
\caption{
TODO
}
\label{tab:modsel}
\end{table}

\begin{figure}
\includegraphics[width=0.9\textwidth]{../../notebook/2021-01-26-indiv-counts-modeling/named-figure/enrichment-5-plots_scz.pdf}
\caption{Enrichment of genomic elements and gene sets in somatic SNVs in schizophrenia}
\label{fig:testresults}
\end{figure}

\section*{Discussion}

Studies of rare germline and de-novo SNVs showed that, although the
genome-wide ``background'' mutation rate is unaltered in schizophrenia,
specific genomic elements and gene sets bear increased burden of---in other
words: are enriched in---SNVs in schizophrenic individuals relative to
controls.  Our present study extends this basic result to somatic SNVs in the
human brain.  In particular, we present evidence that genes implicated in
schizophrenia by schizophrenia GWAS and other psychiatric GWAS suffer
increased rate of somatic point mutations in schizophrenia.

Increased rate suggests that these somatic SNVs are damaging to normal brain
function.  On the other hand we did not observe decreased rate at other
genomic regions, which would allude to compensatory, protective mutations in
schizophrenia.  This observation parallels those regarding rare germline and
de-novo single nucleotide mutations.  Even so, we cannot rule out that
protective somatic SNVs exist in genomic regions mechanistically involved in
schizophrenia so long they are outnumbered by damaging mutations.

The comparison of GWAS genes to GWAS loci suggests that not only 
exonic but also intronic somatic SNVs can contribute to the risk of
schizophrenia.

Questions we cannot answer:

\section*{Methods}

\subsection*{Study subjects and clinical data}

\subsection*{Sample preparation and sequencing}

\subsection*{Somatic SNV calling}

\subsection*{Variant annotation}

SNPnexus (https://www.snp-nexus.org/v4/)

RoadMap,... TODO: list all sources of annotation

\subsection*{Statistical analysis}

For a genomic element or gene set  \(S\) we inspected the joint distribution,
across all study subjects, of disease status Dx, two enrichment variables
\(n_S\), \(f_S\), and technical and biological variables that potentially
carry information on those enrichment variables.  The enrichment variable
\(n_S\) is the number of somatic SNVs located in \(S\), whereas the \(f_S =
	n_S/n_\mathrm{genome}\) is the fraction of somatic SNVs in \(S\)
relative to all somatic SNVs in the genome.  The aim was to formally test
enrichment in \(S\) in schizophrenia by regressing the response variables
\(n_S\) or \(f_S\) on the explanatory variable Dx
and other, ``confounding'', variables.

In case of the plot matrix (Fig~S\ref{fig:scattermatrix}) \(S\) is the set of schizophrenia GWAS genes
defined as all genes, including their introns, that are located fully or
partially in schizophrenia GWAS loci (Table S4 in \cite{Pardinas2018}).

Notable observations:
\begin{itemize}
	\item the strongsest dependence was seen between \(n_S\) and \(n_\mathrm{genome}\) as expected
		when the genome-wide background rate of somatic SNVs is
		substantial relative to enrichment in \(S\)
	\item we have already noted the strong dependence between read depth DP and
		\(n_\mathrm{genome}\)
        \item other dependencies were relatively weak
	\item for many control individuals ageOfDeath is below 50 years while for
		most schizophrenic individuals it is above 50.  This is
		because the Mount Sinai dataset contained older individuals
		that were either control or schizophrenic but the Harvard dataset contained
		only younger control individuals
\end{itemize}

\subsubsection*{Modeling the number and fraction of somatic SNV calls in a gene set}

We considered two generalized linear model families depending on the response
variable:
\begin{description}
\item[response is \(f_S\):] logistic family: binomial error distribution with
	logistic link function; see panel (A)
\item[response is \(n_S\):] log-linear family: Poisson error distribution with exponential link function; see panel (B)
\end{description}

Within each family we considered relatively simple or complex models depending
on the set of variables selected as explanatory.  For \(S =\) schizophrenia
GWAS genes, and within the
logistic family, Fig~S\ref{fig:fitted-lines}A shows fitted lines under both \(f_S \sim 1 +
	\mathrm{Dx}\) and \(f_S \sim 1 + \mathrm{Dx} + \mathrm{ageOfDeath}\)
(panel (A)), where \(1\) means intercept.  For the log-linear family panel
(Fig~S\ref{fig:fitted-lines}B)
shows fitted lines under both \(n_S \sim 1 + \mathrm{Dx}\) and \(n_S \sim 1 +
	\mathrm{Dx} + n_\mathrm{genome}\)


\subsubsection*{Variable selection for regression models}

To select the best fitting model within each family we applied forward
variable selection algorithm with the Akaike information criterion (AIC) as
optimality criterion.  (The Bayesian (BIC) information instead of AIC gave the
same results.)

In the case of \(S =\) schizophrenia GWAS genes we see that the first,
thus most informative, variable selected is Dx under the logistic model
(Fig~S\ref{fig:varsel}A).
Any additional variables increased the information criteria and thus lead to
overfitting the data.

(Fig~S\ref{fig:varsel}B) shows the same result under the log-linear model except that the
most informative variable is \(n_\mathrm{genome}\), which was expected by its
strong correlation with \(n_S\).  Nonetheless, Dx is selected also under this
model family.

\subsubsection*{The effect of ageOfDeath and that of Dataset are mediated by disease status Dx}

As we noted one of our two datasets contained relatively young individuals who
were all control.  This leads to the causal-probabilistic graphical model that
\(\mathrm{ageOfDeath} \leftarrow \mathrm{Dataset} \rightarrow \mathrm{Dx}
\rightarrow y\), where \(y\) is the response enrichment variable \(f_S\) or
\(n_S\).  It means that ageOfDeath and Dataset have indirect effects on
enrichment such that these effects are mediated by disease status Dx. If this
model is true then \(Dx\) d-separates \(y\) from Dataset and ageOfDeath so
that \(y\) becomes independent from ageOfDeath and Dataset if we condition on
Dx = control or Dx = schizophrenia.  Fig~S\ref{fig:d-separation}A illustrates
that.  The important consequence is that we should remove ageOfDeath and
Dataset from the set from our models as long as we include Dx.  Failing to do
so may confound our inference on the effect of Dx on enrichment
(Fig~S\ref{fig:d-separation}B).

\subsubsection*{Residual analysis to assess model fit }

We assessed model fit by residual analysis.  There are several types of residuals in use for generalized linear models.
The $r^\ast$ residual is approximately normally distributed even if the error
distribution for the model family is not normal, such as Poisson or
binomial~\citep{davison2003statistical} (Ch10.2 p477).

This residual is defined as follows

$$
r^\ast = r_\mathrm{D} + r_\mathrm{D}^{-1} \log
\frac{r_\mathrm{P}}{r_\mathrm{D}},
$$

where \(r_\mathrm{D}\) is the standardized deviance residual and
\(r_\mathrm{P}\) is the standardized Pearson residual.

We expressed the size of genomic element or gene set as the number of
individuals with at least one somatic SNV in that genomic element or gene set.
We sorted genomic elements and gene sets according to their size and produced
normal Q-Q plots of \(r^\ast\) from top left of Fig~S\ref{fig:binomial-QQ} (smallest) to
bottom right (largest).  This indicated poor fit for the smallest genomic
elements and gene sets (top row of plots) but good fit for medium sized and
large ones.

\subsubsection*{Multiple test correction}

\subsection*{Data availability}

All FASTQ, CRAM and VCF files generated for this work have been submitted to
collection #2965 of the National Institute of Mental Health Data Archive
(https://nda.nih.gov/) and will be released for the public.

\subsection*{Code availability}

All code developed by A.~Jones and E.~Maury is available at:\\
https://github.com/attilagk/BSM-research

\bibliography{library}


\section*{Acknowledgements}

We thank...

\section*{Author Information}

\section*{Author Contributions}

A.J...

\section*{Competing Interests}

None

\section*{Figure Legends}

\subsection*{Figure 1}
%\includegraphics[width=1.0\textwidth]{figures/by-me/commonmind-rna-seq-ms/commonmind-rna-seq-ms.pdf}

\section*{Tables}

\section*{Supplementary Material}

\setcounter{table}{0}
\makeatletter 
\renewcommand{\figurename}{Supplementary Table} % nice
\makeatother

\setcounter{figure}{0}
\makeatletter 
\renewcommand{\figurename}{Supplementary Figure} % nice
\makeatother

\begin{figure}[p]
\begin{center}
\includegraphics[scale=0.6]{../../notebook/2021-01-26-indiv-counts-modeling/named-figure/symmetric-scatter-matrix_scz.pdf}
\end{center}
\caption{
Pairwise joint distributions of response and explanatory variables TODO: fix legend
}
\label{fig:scattermatrix}
\end{figure}

\begin{figure}[p]
\begin{center}
A\includegraphics[scale=0.45]{../../notebook/2021-01-26-indiv-counts-modeling/named-figure/scz_gwas_genes-ageOfDeath-binom-fit_scz.pdf}
B\includegraphics[scale=0.45]{../../notebook/2021-01-26-indiv-counts-modeling/named-figure/scz_gwas_genes-log10_ncalls-poisson-fit_scz.pdf}
\end{center}
\caption{
Modeling the number (A) and fraction (B) of somatic SNV calls in a gene set
}
\label{fig:fitted-lines}
\end{figure}

\begin{figure}[p]
\begin{center}
A

\includegraphics[scale=0.5]{../../notebook/2021-01-26-indiv-counts-modeling/named-figure/scz_gwas_genes-fw-varsel-binom-onlyIC_scz.pdf}

B

\includegraphics[scale=0.5]{../../notebook/2021-01-26-indiv-counts-modeling/named-figure/scz_gwas_genes-fw-varsel-pois-onlyIC_scz.pdf}
\end{center}
\caption{
Variable selection for regression models. A) Binomial--logistic
family.  B) Poisson--log-linear family.
}
\label{fig:varsel}
\end{figure}

\begin{figure}[p]
\begin{center}
A\includegraphics[scale=0.45]{../../notebook/2021-01-26-indiv-counts-modeling/named-figure/Dx_Dataset_ageOfDeath_scz.pdf}
B\includegraphics[scale=0.45]{../../notebook/2021-01-26-indiv-counts-modeling/named-figure/collinearity_Dx_Dataset_ageOfDeath_scz.pdf}
\end{center}
\caption{
The effect of ageOfDeath and that of Dataset are mediated by disease status Dx
}
\label{fig:d-separation}
\end{figure}

\begin{figure}[p]
\includegraphics[width=0.9\textwidth]{../../notebook/2021-01-26-indiv-counts-modeling/named-figure/resid-binom_scz.pdf}
\caption{
Residual analysis of goodness of fit for the logistic family.
}
\label{fig:binomial-QQ}
\end{figure}

\begin{figure}[p]
\includegraphics[width=0.9\textwidth]{../../notebook/2021-01-26-indiv-counts-modeling/named-figure/resid-pois_scz.pdf}
\caption{
Residual analysis of goodness of fit for the log-linear family.
}
}
\label{fig:poisson-QQ}
\end{figure}

\begin{figure}[p]
\includegraphics[scale=0.6]{../../notebook/2020-07-28-CLOZUK/named-figure/rel-callrate-sczgwas_scz.pdf}
\caption{
TODO
}
\label{fig:scz-gwas-genes-vs-loci}
\end{figure}

\begin{figure}[p]
\includegraphics[scale=0.6]{../../notebook/2020-07-28-CLOZUK/named-figure/calls-in-individual-scz-gwas-loci-coding.pdf}
\caption{
Somatic SNVs at individual loci of schizophrenia GWAS
}
\label{fig:indiv-scz-gwas-loci}
\end{figure}

\end{document}
