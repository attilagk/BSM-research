\documentclass[usenames,dvipsnames]{beamer}
%\documentclass[handout]{beamer}

% language settings
%\usepackage{fontspec, polyglossia}
%\setdefaultlanguage{magyar}

% common packages
\usepackage{amsmath, multimedia, url, hyperref, multirow, colortbl}
\usepackage[table]{xcolor}
%\usepackage{graphicx}

% TikZ
\usepackage{tikz}
%\usetikzlibrary{arrows.meta, decorations.pathmorphing, decorations.pathreplacing, shapes.geometric,mindmap}
%\usetikzlibrary{shapes.geometric,fadings,bayesnet}

% beamer styles
\mode<presentation>{
\usetheme{Singapore}
%\usetheme{Warsaw}
%\usetheme{Antibes}
%\usecolortheme{beaver}
\usecolortheme{crane}
%\usecolortheme{seahorse}
%\usefonttheme{structureitalicserif}
\setbeamercovered{transparent}
}
\setbeamertemplate{blocks}[rounded][shadow=true]
\AtBeginSubsection[]{
  \begin{frame}<beamer>{Contents}
    \tableofcontents[currentsection,currentsubsection]
  \end{frame}
}
%\useoutertheme[]{tree}

% title, etc
\title{Genome-wide Study of Brain Somatic Mosaicism in Schizophrenia}
%\subtitle{A subtitle may be shorter and more technical}
\author{Attila Jones, Chaggai Rosenbluh, Andy Chess}
\date{Brain Somatic Mosaicism Network}

\begin{document}

\maketitle

\section{Genetics of Schizophrenia}

\begin{frame}{What is schizophrenia (SCZ)?}
\begin{columns}[t]
\begin{column}{0.55\textwidth}
	\begin{itemize}
                \item psychotic symptoms
			\begin{itemize}
				\item delusions, hallucinations
			\end{itemize}
                \item dysfunction
			\begin{itemize}
				\item motivational, cognitive
			\end{itemize}
                \item epidemiology
			\begin{itemize}
				\item \(\approx\)1\% prevalence
				\item onset in adolescence
			\end{itemize}
                \item environmental risk
			\begin{itemize}
				\item urban life, migration, drug abuse,
					abuse in childhood
			\end{itemize}
                \item unclear mechanism
			\begin{itemize}
				\item synaptic maturation \& 
					function
                                \item a neurodevelopmental disorder?
			\end{itemize}
	\end{itemize}
\end{column}

\begin{column}{0.45\textwidth}
	genetics carries most risk in SCZ

\includegraphics[width=1.0\columnwidth]{from-others/sullivan2012-fig1a.png}
\end{column}
\end{columns}
%symptoms
%highly heritable
%unclear aetiology:
\end{frame}

\begin{frame}{Shared heritability of psychiatric disorders}
\begin{center}
\includegraphics[height=0.7\textheight]{from-others/brainstorm-consortium-science2018-fig2.jpg}
\end{center}

{\tiny Brainstorm Consortium 2018 Science}
\end{frame}

\begin{frame}{Genetic architecture: polygenic}
\begin{center}
	Polymorphisms (germline variants)

\includegraphics[height=0.8\textheight]{from-others/schizophrenia-kahn-2015-fig3.png}
\end{center}
\end{frame}

\begin{frame}{How to translate genetics to medicine?}
\begin{center}
Lessons from germline variants
\end{center}

\includegraphics[width=1.1\textwidth]{from-others/schizophrenia-Foley2017-Fig1.png}
\end{frame}

\begin{frame}
\begin{columns}[t]
\begin{column}{0.3\textwidth}
\begin{center}
Germline variant
\end{center}
\end{column}

\begin{column}{0.7\textwidth}
\begin{center}
Somatic variant
\end{center}
\end{column}
\end{columns}
\begin{columns}[t]
\begin{column}{0.3\textwidth}
\begin{center}
	\footnotesize no mosaicism
\end{center}

\includegraphics[width=1.0\columnwidth]{from-others/poduri-2013-fig1a.png}
\begin{center}
\end{center}
\end{column}

\begin{column}{0.3\columnwidth}
\begin{center}
	\footnotesize high freq.~mosaicism
\end{center}

\includegraphics[width=1.0\columnwidth]{from-others/poduri-2013-fig1e.png}
\end{column}
\begin{column}{0.3\columnwidth}
\begin{center}
	\footnotesize low freq.~mosaicism
\end{center}

\includegraphics[width=1.0\columnwidth]{from-others/poduri-2013-fig1g.png}
\end{column}
\end{columns}
{\tiny Poduri et al 2013 Science}
\end{frame}

\section{Establishing Somatic Variant Calling} 
\begin{frame}
\begin{columns}[t]
\begin{column}{0.6\textwidth}

\includegraphics[width=1.0\columnwidth]{from-others/bsmn-visual-summary.jpg}

{\tiny Brain Somatic Mosaicism Network 2017 Science}
\end{column}

\begin{column}{0.4\textwidth}
\begin{itemize}
\item 7 NIH projects
	\begin{itemize}
	\item sequence brain cells/samples
	\item call somatic variants
	\end{itemize}
\item collaboration
	\begin{itemize}
	\item methods \& tools 
        \item validation
        \item functional experiments
	\end{itemize}
\item public data
	\begin{itemize}
		\item NIMH Data Archive
	\end{itemize}
\end{itemize}
\end{column}
\end{columns}
\end{frame}

\begin{frame}{Calling somatic variants}
\includegraphics[width=1.0\textwidth]{from-others/bsm-marker-fig1.jpg}

{\tiny Brain Somatic Mosaicism Network 2017 Science}
\end{frame}

\begin{frame}{Traditional variant calling practices failed}
\includegraphics[height=0.8\textheight]{from-others/bsmn-best-practices-fig2ab.png}

{\tiny Brain Somatic Mosaicism Network 2020 bioRxiv}
\end{frame}

\begin{frame}{Validation results suggested new best practices}
\begin{center}
\includegraphics[height=0.7\textheight]{from-others/bsmn-best-practices-fig2c.png}
\end{center}

{\tiny Brain Somatic Mosaicism Network 2020 bioRxiv}
\end{frame}

\begin{frame}{Best practices variant calling pipeline}
\includegraphics[width=1.0\textwidth]{from-others/bsmn-pipeline-calling.png}
\begin{center}
\begin{itemize}
\item software tool
\begin{itemize}
        \item cloud computing: compute resources auto-scale with data
	\item  \url{https://github.com/bsmn/bsmn-pipeline}
\end{itemize}
\item validation rate (specificity): 85\%
\item what about sensitivity?
\end{itemize}
\end{center}

{\tiny Brain Somatic Mosaicism Network 2020 bioRxiv}
\end{frame}

\begin{frame}[label=sensitivity]{Sensitivity of variant calling pipeline}{depends on sequencing depth and allele frequency}
\begin{columns}[t]
\begin{column}{0.35\textwidth}
\begin{center}
\footnotesize
toy data, ``labeled'' variants
\end{center}

\includegraphics[width=1.0\textwidth]{from-others/bsmn-best-practices-fig4b.png}

{\tiny BSM Network 2020 bioRxiv}
\end{column}

\begin{column}{0.65\textwidth}
\begin{center}
\footnotesize
real data, ``unlabeled'' variants
\end{center}

\includegraphics<1>[width=1.0\textwidth]{../../notebook/2020-08-13-select-vars/named-figure/DP-AF-jointdensity-calls.pdf}

\includegraphics<2>[width=1.0\textwidth]{../../notebook/2020-08-13-select-vars/named-figure/DP-AF-jointdensity-conddensity.pdf}

{\tiny A.~Jones, unpublished }
\end{column}
\end{columns}
\end{frame}

\section{Somatic Variants in Schizophrenia}

\begin{frame}{Somatic variants in schizophrenia (SCZ)}
	Outline
\begin{enumerate}
	\item single nucleotide mutations
		\begin{itemize}
			\item genom wide rate
			\item general functinal categories
			\item schizophrenia GWAS loci
		\end{itemize}
        \item copy number variants
		\begin{itemize}
			\item variant calling in progress \\
				\url{github.com/freeseek/mocha}
			\item Eduardo Maury (Harvard)
		\end{itemize}
\end{enumerate}
\end{frame}

\begin{frame}{Somatic single nucleotide variant calls}
\begin{columns}[t]
\begin{column}{0.7\textwidth}

\includegraphics[scale=0.5]{../../notebook/2020-06-10-annotate-explore-variants/named-figure/num-indiv-calls-Dx.pdf}
\end{column}

\begin{column}{0.5\textwidth}

\includegraphics[scale=0.5]{../../notebook/2020-06-10-annotate-explore-variants/named-figure/numcalls-Dx.pdf}
\end{column}
\end{columns}
\end{frame}

\againframe<2>{sensitivity}

\begin{frame}[label=regr-models]{Modeling the joint effect of SCZ and other features}
\tiny
\begin{tabular}{cc|c|ccl}
	& & & & clinical feat. & technical \& genomic features \\
sample & variant & \(\mathrm{\#calls}\) & Dx & \(X\) & \(a(Y)\) \\
\hline
\multirow{4}{*}{\colorbox{Apricot}{\(s_1\)}} & \colorbox{Yellow}{\(v_1\)} & \multirow{4}{*}{\(4\)} & \multirow{4}{*}{SCZ/1} & \multirow{4}{*}{\colorbox{Apricot}{\(x_1\)}} &
\multirow{4}{*}{\(\colorbox{Apricot}{\(a_1\)} = a(\colorbox{Yellow}{\(y_1\)},
		\colorbox{Periwinkle}{\(y_2\)},
		\colorbox{CarnationPink}{\(y_3\)},
		\colorbox{CornflowerBlue}{\(y_4\)})\)} \\
& \colorbox{Periwinkle}{\(v_2\)} & & \\
& \colorbox{CarnationPink}{\(v_3\)} & & \\
& \colorbox{CornflowerBlue}{\(v_4\)} & & \\
\hline
\multirow{2}{*}{\colorbox{LimeGreen}{\(s_2\)}} & \colorbox{Salmon}{\(v_5\)} & \multirow{2}{*}{\(2\)} & \multirow{2}{*}{SCZ/1} & \multirow{2}{*}{\colorbox{LimeGreen}{\(x_2\)}} &
\multirow{2}{*}{\(\colorbox{LimeGreen}{\(a_2\)} = a(\colorbox{Salmon}{\(y_5\)}, \colorbox{Tan}{\(y_6\)})\)} \\
& \colorbox{Tan}{\(v_6\)} & & \\
\hline
\multirow{3}{*}{\colorbox{Lavender}{\(s_3\)}} & \colorbox{Dandelion}{\(v_7\)} & \multirow{4}{*}{\(3\)} & \multirow{4}{*}{Control/0} & \multirow{3}{*}{\colorbox{Lavender}{\(x_3\)}} &
\multirow{3}{*}{\(\colorbox{Lavender}{\(a_3\)} = a(\colorbox{Dandelion}{\(y_7\)}, \colorbox{Emerald}{\(y_8\)}, \colorbox{Gray}{\(y_9\)})\)} \\
& \colorbox{Emerald}{\(v_8\)} & & \\
& \colorbox{Gray}{\(v_9\)} & & \\
\end{tabular}
%dental consultation => oral surgery program for authorization
%212 423 4400
%oral surgery appointment
%212 241 7131
%212 241 4148
\vfill
\normalsize

\[\mathrm{\#calls} =
	\overbrace{\mathrm{Dx}\mathbf{\beta}_\mathrm{Dx}}^{\text{effect of
	schizophrenia}} + X\beta_x + a(Y)\beta'_y + \epsilon\]
\end{frame}

\begin{frame}{Unaltered mutation rate in schizophrenic brains...}
\begin{columns}[t]
\begin{column}{0.5\textwidth}

\includegraphics[width=1.0\textwidth]{from-others/forward-stepwise-algorithm.png}
\end{column}

\begin{column}{0.5\textwidth}
\begin{center}
\small
selected 15 features (out of 48)
\end{center}
\tiny
\begin{tabular}{rcc}
	feature/variable & \(P>|t|\) & sign of effect \\
\hline
DP std & 0.007 & + \\
AF mean & 0.011 & - \\
culprit entropy & 0.011 & + \\
ReportedGender[T.Male] & 0.016 & - \\
AntipsychAtyp[T.1] & 0.033 & + \\
Anticonvulsants[T.1] & 0.043 & - \\
REF entropy & 0.054 & + \\
szdbCNVcount std & 0.065 & + \\
YearofAutopsy & 0.069 & - \\
Intercept & 0.114 & + \\
EV1 & 0.147 & - \\
DP mean & 0.183 & + \\
Alcohol[T.1] & 0.252 & - \\
Lithium[T.1] & 0.284 & - \\
EV3 & 0.292 & + \\
Ethnicity[T.Caucasian] & 0.498 & - \\
Ethnicity[T.Hispanic] & 0.897 & - \\
\end{tabular}
\end{column}
\end{columns}
\end{frame}

\begin{frame}<1>[label=outlier]{...except one brain} 
\begin{columns}[t]
\begin{column}{0.6\textwidth}

\includegraphics[width=\columnwidth]{../../notebook/2020-08-19-indiv-based-analysis/named-figure/numcalls-depth-Dx.pdf}
\end{column}

\begin{column}{0.5\textwidth}

\begin{center}
the outlier
\end{center}
\begin{itemize}
		\footnotesize
	\item schizophrenic female
	\item<2> deleterious mutations in NOTCH3 and WFIKKN1 (a protease inhibitor)
        \item<2> NOTCH3 implicated in
		\begin{itemize}
			\item neural development
			\item hereditary stroke, lateral meningocele syndrome
		\end{itemize}
	\item<2> but why the high mutation rate?
\end{itemize}
\end{column}
\end{columns}
\end{frame}

\begin{frame}[label=annotations]{Annotating variants with genomic features}
\tiny

%\rowcolors{1}{green}{pink}
\begin{tabular}{rp{2in}}
\scshape{Category} & \scshape{Details}\\
\hline
&\\
\multicolumn{2}{l}{\textit{General}} \\
\hline
\rowcolor{Yellow}
Gene Annotation & Ensembl\\
\rowcolor{Dandelion}
Protein Effect & SIFT, PolyPhen\\
\rowcolor{Yellow}
Population Data & 1000 Genomes, gnomAD\\
\rowcolor{Dandelion}
Regulatory Elements & TFBS, miRBASE, Vista, CpG Islands, TargetScan, TarBase miRNA, Other RNAs, ENCODE regions, RoadMap Epigenomics, Ensembl Regulatory Build\\
\rowcolor{Yellow}
Phenotype/Disease Association & GAD, COSMIC, GWAS, ClinVar\\
\rowcolor{Dandelion}
Conserved Elements & PhastConsElements, GERP++\\
\rowcolor{Yellow}
Structural Variations &\\
\rowcolor{Dandelion}
Non-coding Variation Scoring & DeepSEA\\
\rowcolor{Yellow}
Reactome Pathways &\\
\rowcolor{Dandelion}
Cancer Genome Interpreter &\\
\hline
&\\
\multicolumn{2}{l}{\textit{Brain specific}} \\
\hline
\rowcolor{Yellow}
Chromatin state & RoadMap Epigenomics \\
\hline
&\\
\multicolumn{2}{l}{\textit{Scizophrenia specific}} \\
\hline
\rowcolor{Yellow}
SCZ GWAS & susceptibility loci \\
\rowcolor{Dandelion}
SCZ copy number variants & literature weighted\\
\hline
\end{tabular}
\end{frame}

\againframe<2>{outlier}

\begin{frame}{General functional categories are not enriched in SCZ variants}
\small
\begin{tabular}{r|cc}
& \multicolumn{2}{c}{\# variant calls} \\
& Control & SCZ \\
\rowcolor{gray}
total & 901 & 2510 \\
& & \\
\scshape{functional category} & & \\
\rowcolor{SkyBlue}
coding nonsyn & 15 {\tiny (16)} & 44 {\tiny (43)} \\
\rowcolor{Turquoise}
stop-gain & 0 {\tiny (1)} & 4 {\tiny (3)}\\
\rowcolor{SkyBlue}
intronic (splice site) & 0 {\tiny (0)}& 1 {\tiny (1)}\\
\rowcolor{Turquoise}
in nervous sys epigenome & 208 {\tiny (193)}& 521 {\tiny (536)}\\
\rowcolor{SkyBlue}
transcription factor binding site & 22 {\tiny (27)}& 79 {\tiny (74)}\\
\rowcolor{Turquoise}
deleterious & 6 {\tiny (5)}& 14 {\tiny (15)}\\
\rowcolor{SkyBlue}
deleterious - low confidence & 1 {\tiny (1)}& 3 {\tiny (3)}\\
\rowcolor{Turquoise}
non-coding, predicted functional & 3 {\tiny (4)}& 12 {\tiny (11)}\\
\rowcolor{SkyBlue}
in evolutionary conserved element & 75 {\tiny (90)}& 265 {\tiny (250)}\\
\end{tabular}
\end{frame}

\againframe{annotations}

\begin{frame}{No chromatin state is enriched in SCZ variants}
\includegraphics[scale=0.7]{../../notebook/2020-06-10-annotate-explore-variants/named-figure/chromatin-state-DLPFC-Dx-10.pdf}
\end{frame}

\begin{frame}{SCZ susceptibility loci are not enriched in SCZ variants}
\includegraphics[scale=0.7]{../../notebook/2020-06-10-annotate-explore-variants/named-figure/gwas-Dx.pdf}
\end{frame}

\begin{frame}{Enrichment expected in core genes}
\includegraphics[height=0.7\textheight]{from-others/polygenic-to-omnigenic-fig4a.jpg}

{\tiny Boyle et al 2017 Cell}
\end{frame}

\begin{frame}{Planned analyses}
test
\begin{itemize}
\item genes enriched in rare or de novo variants in SCZ
\item Gene Ontology categories related to SCZ
\begin{itemize}
\item synaptic function
\end{itemize}
\end{itemize}

\vfill
discover
\begin{itemize}
\item genes with recurrent variants
\item Gene Ontology entities enriched in SCZ \textit{vs} Control
\end{itemize}
\end{frame}

\begin{frame}{Summary}
\begin{enumerate}
\item Schizophrenia is a polygenic disease
	\begin{itemize}
                \item unclear pathological mechanism
		\item comorbidity and shared heritability with psychiatric and
			neurodevelopmental disorders
	\end{itemize}
\item We study somatic variants in schizophrenia
	\begin{itemize}
		\item The Brain Somatic Mosaicism Network 
		\item new tool: somatic single nucleotide variant (SNV) calling pipeline
	\end{itemize}
\item Somatic SNV mutation rate in schizophrenia is unaltered:
	\begin{itemize}
		\item genome-wide
		\item in general functinal categories
                \item in schizophrenia GWAS loci
	\end{itemize}
\item Ongoing: more specific investigation of genes and mechanisms
\end{enumerate}
\end{frame}

\begin{frame}{Acknowledgements}
\footnotesize
\begin{tabular}{cp{3in}}
\parbox[c]{9em}{\includegraphics[width=0.25\textwidth]{/home/attila/projects/bsm/resources/logos/mount-sinai-logo.png}}
& Chaggai Rosenbluh, Gabriel Hoffman, Eva Xia, Jessica Johnson, Schahram Akbarian, Andy Chess \\
\noalign{\medskip}
\parbox[c]{9em}{\includegraphics[width=0.3\textwidth]{/home/attila/projects/bsm/resources/logos/mayo-clinic-logo.png}} & Taejeong Bae, Alexej Abyzov \\
\noalign{\medskip}
\parbox[c]{9em}{\includegraphics[width=0.3\textwidth]{/home/attila/projects/bsm/resources/logos/harvard-logo.png}}
& Eduardo Maury, Yanmei Dou, Peter Park, Chris Walsh \\
\noalign{\medskip}
\parbox[c]{9em}{\includegraphics[width=0.3\textwidth]{/home/attila/projects/bsm/resources/logos/sage-bionetworks-logo.png}} & Kenny Daily, Cindy Molitor, Mette Peters \\
\noalign{\medskip}
\parbox[c]{9em}{\includegraphics[width=0.3\textwidth]{/home/attila/projects/bsm/resources/logos/nimh-logo.png}}
& David Obenshain \\
\noalign{\medskip}
\parbox[c]{9em}{\includegraphics[width=0.3\textwidth]{/home/attila/projects/bsm/resources/logos/BSMN-NameAndTagline-Horizontal.png}}
& \\
\end{tabular}
\end{frame}

\end{document}

\begin{columns}[t]
\begin{column}{0.5\textwidth}

\end{column}

\begin{column}{0.5\textwidth}

\end{column}
\end{columns}
