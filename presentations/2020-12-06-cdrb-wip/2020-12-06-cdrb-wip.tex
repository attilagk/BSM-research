\documentclass[usenames,dvipsnames]{beamer}
%\documentclass[handout]{beamer}

% language settings
%\usepackage{fontspec, polyglossia}
%\setdefaultlanguage{magyar}

% common packages
\usepackage{amsmath, multimedia, url, hyperref, xcolor, multirow}
%\usepackage{graphicx}

% TikZ
\usepackage{tikz}
%\usetikzlibrary{arrows.meta, decorations.pathmorphing, decorations.pathreplacing, shapes.geometric,mindmap}
%\usetikzlibrary{shapes.geometric,fadings,bayesnet}

% beamer styles
\mode<presentation>{
\usetheme{Singapore}
%\usetheme{Warsaw}
%\usetheme{Antibes}
%\usecolortheme{beaver}
\usecolortheme{crane}
%\usecolortheme{seahorse}
%\usefonttheme{structureitalicserif}
\setbeamercovered{transparent}
}
\setbeamertemplate{blocks}[rounded][shadow=true]
\AtBeginSubsection[]{
  \begin{frame}<beamer>{Contents}
    \tableofcontents[currentsection,currentsubsection]
  \end{frame}
}
%\useoutertheme[]{tree}

% title, etc
\title{Somatic SNVs from WGS data in Schizophrenic Brains}
%\subtitle{A subtitle may be shorter and more technical}
\author{Attila Jones, Chaggai Rosenbluh, Andy Chess}
\date{Brain Somatic Mosaicism Network}

\begin{document}

\maketitle

\section{Genetics of Schizophrenia}

\begin{frame}{What is schizophrenia (SCZ)?}
\begin{columns}[t]
\begin{column}{0.55\textwidth}
	\begin{itemize}
                \item psychotic symptoms
			\begin{itemize}
				\item delusions, hallucinations
			\end{itemize}
                \item dysfunction
			\begin{itemize}
				\item motivational, cognitive
			\end{itemize}
                \item epidemiology
			\begin{itemize}
				\item \(\approx\)1\% prevalence
				\item onset in adolescence
			\end{itemize}
                \item environmental risk
			\begin{itemize}
				\item urban life, migration, drug abuse,
					abuse in childhood
			\end{itemize}
                \item unclear mechanism
			\begin{itemize}
				\item synaptic maturation \& 
					function
                                \item a neurodevelopmental disorder?
			\end{itemize}
	\end{itemize}
\end{column}

\begin{column}{0.45\textwidth}
	genetics carries most risk in SCZ

\includegraphics[width=1.0\columnwidth]{from-others/sullivan2012-fig1a.png}
\end{column}
\end{columns}
%symptoms
%highly heritable
%unclear aetiology:
\end{frame}

\begin{frame}{Shared heritability of psychiatric disorders}
\begin{center}
\includegraphics[height=0.7\textheight]{from-others/brainstorm-consortium-science2018-fig2.jpg}
\end{center}

{\tiny Brainstorm Consortium 2018 Science}
\end{frame}

\begin{frame}{Genetic architecture: polygenic}
\begin{center}
	Polymorphisms (germline variants)

\includegraphics[height=0.8\textheight]{from-others/schizophrenia-kahn-2015-fig3.png}
\end{center}
\end{frame}

\begin{frame}{How to translate genetics to medicine?}
\begin{center}
Lessons from germline variants
\end{center}

\includegraphics[width=1.1\textwidth]{from-others/schizophrenia-Foley2017-Fig1.png}
\end{frame}

\begin{frame}
\begin{columns}[t]
\begin{column}{0.3\textwidth}
\begin{center}
Germline variant
\end{center}
\end{column}

\begin{column}{0.7\textwidth}
\begin{center}
Somatic variant
\end{center}
\end{column}
\end{columns}
\begin{columns}[t]
\begin{column}{0.3\textwidth}
\begin{center}
	\footnotesize no mosaicism
\end{center}

\includegraphics[width=1.0\columnwidth]{from-others/poduri-2013-fig1a.png}
\begin{center}
\end{center}
\end{column}

\begin{column}{0.3\columnwidth}
\begin{center}
	\footnotesize high freq.~mosaicism
\end{center}

\includegraphics[width=1.0\columnwidth]{from-others/poduri-2013-fig1e.png}
\end{column}
\begin{column}{0.3\columnwidth}
\begin{center}
	\footnotesize low freq.~mosaicism
\end{center}

\includegraphics[width=1.0\columnwidth]{from-others/poduri-2013-fig1g.png}
\end{column}
\end{columns}
{\tiny Poduri et al 2013 Science}
\end{frame}

\section{Establishing Somatic Variant Calling} 
\begin{frame}
\begin{columns}[t]
\begin{column}{0.6\textwidth}

\includegraphics[width=1.0\columnwidth]{from-others/bsmn-visual-summary.jpg}

{\tiny Brain Somatic Mosaicism Network 2017 Science}
\end{column}

\begin{column}{0.4\textwidth}
\begin{itemize}
\item 7 NIH projects
	\begin{itemize}
	\item sequence brain cells/samples
	\item call somatic variants
	\end{itemize}
\item collaboration
	\begin{itemize}
	\item methods \& tools 
        \item validation
        \item functional experiments
	\end{itemize}
\item public data
	\begin{itemize}
		\item NIMH Data Archive
	\end{itemize}
\end{itemize}
\end{column}
\end{columns}
\end{frame}

\begin{frame}{Calling somatic variants}
\includegraphics[width=1.0\textwidth]{from-others/bsm-marker-fig1.jpg}

{\tiny Brain Somatic Mosaicism Network 2017 Science}
\end{frame}

\begin{frame}{Traditional variant calling practices failed}
\includegraphics[height=0.8\textheight]{from-others/bsmn-best-practices-fig2ab.png}

{\tiny Brain Somatic Mosaicism Network 2020 bioRxiv}
\end{frame}

\begin{frame}{Validation results suggested new best practices}
\begin{center}
\includegraphics[height=0.7\textheight]{from-others/bsmn-best-practices-fig2c.png}
\end{center}

{\tiny Brain Somatic Mosaicism Network 2020 bioRxiv}
\end{frame}

\begin{frame}{Best practices variant calling pipeline}
\includegraphics[width=1.0\textwidth]{from-others/bsmn-pipeline-calling.png}
\begin{center}
\begin{itemize}
\item software tool
\begin{itemize}
        \item cloud computing: compute resources auto-scale with data
	\item  \url{https://github.com/bsmn/bsmn-pipeline}
\end{itemize}
\item validation rate (specificity): 85\%
\item what about sensitivity?
\end{itemize}
\end{center}

{\tiny Brain Somatic Mosaicism Network 2020 bioRxiv}
\end{frame}

\begin{frame}{Sensitivity of variant calling pipeline}{depends on sequencing depth and allele frequency}
\begin{columns}[t]
\begin{column}{0.35\textwidth}
\begin{center}
\footnotesize
toy data, ``labeled'' variants
\end{center}

\includegraphics[width=1.0\textwidth]{from-others/bsmn-best-practices-fig4b.png}

{\tiny BSM Network 2020 bioRxiv}
\end{column}

\begin{column}{0.65\textwidth}
\begin{center}
\footnotesize
real data, ``unlabeled'' variants
\end{center}

\includegraphics<1>[width=1.0\textwidth]{../../notebook/2020-08-13-select-vars/named-figure/DP-AF-jointdensity-calls.pdf}

\includegraphics<2>[width=1.0\textwidth]{../../notebook/2020-08-13-select-vars/named-figure/DP-AF-jointdensity-conddensity.pdf}

{\tiny A.~Jones, unpublished }
\end{column}
\end{columns}
\end{frame}

\section{Somatic Variants in Schizophrenia}

\begin{frame}{Somatic variant calls}

\end{frame}

\begin{frame}{Genomic annotations}

\end{frame}

\begin{frame}
\includegraphics[height=0.7\textheight]{from-others/polygenic-to-omnigenic-fig4a.jpg}

{\tiny Boyle et al 2017 Cell}
\end{frame}

\end{document}

\begin{columns}[t]
\begin{column}{0.5\textwidth}

\end{column}

\begin{column}{0.5\textwidth}

\end{column}
\end{columns}
