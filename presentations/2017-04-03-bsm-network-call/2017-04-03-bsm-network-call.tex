\documentclass{beamer} %\documentclass[handout]{beamer}


%\includeonlyframes{bsm,title,toc,imprint-mouse-devel,igf2-imprint-evol,sister-disorders,previous-age-studies,our-study,cmc,toc-current,filtering-calling,fitting-models,ll-surface,all-betas,orthogonality,identifiability,anova,betas-cluster,signif-genes,cf-overall-expression,summary,improving-model,p-val,imprinted-brain,chess-lab}

% language settings
%\usepackage{fontspec, polyglossia}
%\setdefaultlanguage{magyar}

% common packages
\usepackage{amsmath, multimedia, hyperref, color, multirow}
%\usepackage{graphicx}

% TikZ
\usepackage{tikz}
%\usetikzlibrary{arrows.meta, decorations.pathmorphing, decorations.pathreplacing, shapes.geometric,mindmap}
%\usetikzlibrary{shapes.geometric,fadings,bayesnet}

% beamer styles
\mode<presentation>{
\usetheme{Boadilla}
%\usetheme{Copenhagen}
%\usecolortheme{beaver}
\usecolortheme{default}
%\usefonttheme{structureitalicserif}
\setbeamercovered{transparent}
}
\setbeamertemplate{blocks}[rounded][shadow=true]
\AtBeginSection[]{
  \begin{frame}<beamer>{Contents}
    %\tableofcontents[currentsection]
    \tableofcontents[currentsection]
  \end{frame}
}
%\useoutertheme[]{tree}

\title{Accurate variant calling with FDR control}
\author{Attila Guly\'{a}s-Kov\'{a}cs}
\date{Chess Lab at Mount Sinai}

\newcommand{\platefigscale}[0]{0.7}
\newcommand{\ownfigscale}[0]{0.4}

\begin{document}

\begin{frame}[plain, label=title]
\maketitle
\end{frame}

\section{The problem}

\begin{frame}
posterior evidence for the role of gene \(g\) in a phenotype

prior evidence

+ \(\int_{v\in g}\) evidence
\end{frame}

\begin{frame}{}

We have:

\begin{tabular}{cl}
%\(\mathcal{T}\) & true variants \\
%\(v\) & a called variant \\
%\(\mathcal{C}\) & called variants \\
\(\mathcal{V}\) & true (but unknown) variants \\
\(\{E_v : v \in \mathcal{V} \} \) & prior evidence that true variant \(v\) is causal \\
\(D\) & sequence data \\
\end{tabular}

\bigskip

Suppose \(i\) was called a variant.
\begin{itemize}
\item Is \(i\) a true variant?
\item Is \(i\) also causal?
\end{itemize}

\bigskip

\begin{eqnarray*}
P(i \text{ is true and causal } | D, E_i, i \text{ called})
&=&
\overbrace{P(i \text{ is true } | D, i \text{ called})}^{\pi_i} \\
&\times&
P(i \text{ is causal } | i \text{ is true}, E_i)
\end{eqnarray*}
\end{frame}

\begin{frame}{FDR and precision}
\begin{eqnarray*}
\text{FDR} &=& \text{E} \left[ \frac{\text{\# false calls}}{\# calls} \right] \\
&=& 1 - \underbrace{\text{E} \left[ \frac{\text{\# true calls}}{\# calls}
\right]}_{\text{precision or PPV}}
\end{eqnarray*}
\end{frame}

\begin{frame}{How FDR control works}

\begin{enumerate}
\item set FDR to some desired level \(\alpha\)
\item rank candidate variants by \(\pi_i\) 
\item for each rank \(j\) calculate
\begin{equation*}
\overbrace{\text{FDR}_{(j)}}^{q_{(j)} \text{-value}} = 1 - \text{precision}_{(j)} = 1 -
%\frac{1}{j} \sum_{k=1}^j \pi_{(k)}
\frac{\pi_{(1)} + ... + \pi_{(j)}}{j}
\end{equation*}
\item call the \(j'\) highest ranking candidates if \(\text{FDR}_{(j')} <
\alpha < \text{FDR}_{(j'+1)}\)

\end{enumerate}

\bigskip
{\footnotesize
\begin{tabular}{cl}
\hline
%\(j\) & the rank of candidate variant \(i\) \\
\(\text{FDR}_{(j)}\) & the minimum FDR calling the \(j\) highest ranking candidates \\
\(\pi_i\) & the probability that \(i\) is a true call \\
\end{tabular}
}
\end{frame}

\begin{frame}{The well-known problem: calls depend on callers}
\begin{columns}[t]
\begin{column}{0.5\textwidth}

\includegraphics[width=1.0\columnwidth]{figures/from-others/kim-2013-bmcbioinf-fig1a.png}
\end{column}
\begin{column}{0.5\textwidth}

\includegraphics[width=1.0\columnwidth]{figures/from-others/kim-2013-bmcbioinf-fig1b.png}
\end{column}
\end{columns}
\end{frame}

\section{A solution: VariantMetaCaller (VMC)}

\begin{frame}{Combining variants with BAYSIC}
{BMC Bioinformatics. 2014; 15: 104.}

\includegraphics[height=0.7\textheight]{figures/from-others/baysic-fig6.png}
\end{frame}

\begin{frame}{VariantMetaCaller (VMC)}
{G\'{e}zsi,..., Antal, BMC Genomics. 2015; 16: 875.}

\includegraphics[width=0.9\textwidth]{figures/from-others/vmc-fig1.png}
\end{frame}

\begin{frame}{Performance and precision}
\begin{columns}[t]
\begin{column}{0.5\textwidth}
performance

\includegraphics[height=0.8\textheight]{figures/from-others/vmc-fig4b-bwa.png}
\end{column}

\begin{column}{0.5\textwidth}
error of precision

\includegraphics[height=0.8\textheight]{figures/from-others/vmc-fig4c-bwa.png}
\end{column}
\end{columns}
\end{frame}

%\begin{frame}{Brain Somatic Mosaicism}
%\begin{columns}[t]
%\begin{column}{0.6\textwidth}
%
%\includegraphics[width=1.0\columnwidth]{figures/from-others/gage-curropsysbio-2016-1.jpg}
%
%{\tiny Paquola, Erwin, Gage 2016}
%\end{column}
%
%\begin{column}{0.4\textwidth}
%challenges with somatic variants:
%\begin{enumerate}
%\item detection\\
%{\footnotesize allelic fraction}
%\item prioritization\\{\footnotesize multiple info}
%\item integration\\{\footnotesize germline vars.}
%\end{enumerate}
%\end{column}
%\end{columns}
%\end{frame}

\end{document}
