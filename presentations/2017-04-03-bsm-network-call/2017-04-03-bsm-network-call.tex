\documentclass{beamer} %\documentclass[handout]{beamer}


%\includeonlyframes{bsm,title,toc,imprint-mouse-devel,igf2-imprint-evol,sister-disorders,previous-age-studies,our-study,cmc,toc-current,filtering-calling,fitting-models,ll-surface,all-betas,orthogonality,identifiability,anova,betas-cluster,signif-genes,cf-overall-expression,summary,improving-model,p-val,imprinted-brain,chess-lab}

% language settings
%\usepackage{fontspec, polyglossia}
%\setdefaultlanguage{magyar}

% common packages
\usepackage{amsmath, multimedia, hyperref, color, multirow}
%\usepackage{graphicx}

% TikZ
\usepackage{tikz}
%\usetikzlibrary{arrows.meta, decorations.pathmorphing, decorations.pathreplacing, shapes.geometric,mindmap}
%\usetikzlibrary{shapes.geometric,fadings,bayesnet}

% beamer styles
\mode<presentation>{
%\usetheme{Pittsburgh}
\usetheme{Copenhagen}
\usecolortheme{beaver}
%\usecolortheme{seahorse}
%\usefonttheme{structureitalicserif}
\setbeamercovered{transparent}
}
\setbeamertemplate{blocks}[rounded][shadow=true]
\AtBeginSection[]{
  \begin{frame}<beamer>{Contents}
    %\tableofcontents[currentsection]
    \tableofcontents[currentsection]
  \end{frame}
}
%\useoutertheme[]{tree}

\title{Work in the Chess lab at MSSM}
\author{Attila Guly\'{a}s-Kov\'{a}cs}
\date{Mount Sinai School of Medicine}

\newcommand{\platefigscale}[0]{0.7}
\newcommand{\ownfigscale}[0]{0.4}

\begin{document}

\begin{frame}[plain, label=title]
\maketitle
\end{frame}

\section{Imprinted genes in the brain}

%\subsection{Imprinting and allelic bias}

\begin{frame}[label=chess-lab]{The Chess lab}
\begin{tabular}{r|l}
Andy Chess & BSM, replication, imprinting, drop-seq \\
Chaggai Rosenbluh & BSM, replication \\
Attila Guly\'{a}s-Kov\'{a}cs & BSM, imprinting \\
Mehaa Bajaj & BSM, replication, imprinting, drop-seq \\
Eva Xia & drop-seq, imprinting \\
\end{tabular}
\end{frame}

\begin{frame}{Imprinting and allelic bias}
\begin{enumerate}
\item epigenetic mechanism
\item variation across age and tissue
\item biological function 
\end{enumerate}
\includegraphics[width=0.6\textwidth]{figures/from-others/renfree-2012-fig2.jpg}

{\tiny Renfree et al 2012 Philos Trans R Soc Lond B}

\end{frame}

%\begin{frame}{Increasing evolutionary prevalence}{}
%{\tiny Smits et al 2008 Nat Genet}
%
%\includegraphics[width=0.7\textwidth]{figures/from-others/smits-et-al-2008-fig5.jpg}
%
%\begin{flushright}
%How many imprinted genes?
%
%\(\approx 100 \; \leftrightarrow \; \approx 1300\) \footnote{Gregg et al 2010}
%\end{flushright}
%\end{frame}

%% shift away from normal allelic bias in two possible directions => sister disorders
%% general: growth disorders, mental defects
%% Igf2/H19: Beckwith–Wiedeman syn, Silver-Russell syn
%% Ube3a cluster 15q13: Angelman (mental retardation); Prader-Willi (obesity, mental defects)
%% CNV in 15q13 increases risk of SCZ, a highly polygenic disorder with many common variants with mild risk
%\begin{frame}[t, label=sister-disorders]{Dysfunction: development and growth}
%\includegraphics[width=0.85\textwidth]{figures/from-others/peters-2014-imprinting-fig1b.jpg}
%{\tiny Peters 2014}
%\begin{columns}[t]
%\begin{column}{0.5\textwidth}
%{\tiny Angelman syndrome. Boy with a puppet}
%
%\includegraphics<1>[width=0.60\columnwidth]{figures/from-others/boy-with-a-puppet-Giovanni-Francesco-Caroto.jpg}
%
%%{\tiny Boy with a Puppet}
%\end{column}
%\begin{column}{0.5\textwidth}
%{\tiny Prader-Willi syndr.  Eugenia ``La Monstrua''}
%
%\includegraphics<1>[width=0.70\columnwidth]{figures/from-others/Eugenia-Martínez-Vallejo-clothed-cropped.jpg}
%
%%{\tiny Eugenia Mart\'{i}nez Vallejo}
%\end{column}
%\end{columns}
%
%\end{frame}
%
%\begin{frame}{Dysfunction: psychology}{Angelman/Prader-Willi region implicated
%in schizophrenia}
%\includegraphics[width=1.0\textwidth]{figures/from-others/sullivan-natrevgenet-2012-fig1b.jpg}
%
%{\tiny Sullivan 2012 Nat Rev Genet.}
%\end{frame}

%% uses kinship theory to explain psychiatric conditions
%% PEGs self-oriented cognition, - inc. fitness, autistic; MEGs mentalistic cognition, + incl. fitness, psychotic
%% this hypothesis guides the study the role of imprinting in psychiatric disorders e.g. SCZ
%\begin{frame}[label=imprinted-brain]{The imprinted brain theory}
%\begin{center}
%\includegraphics[width=0.7\textwidth]{figures/from-others/crespi-2008-fig3.png}
%\vfill
%{\tiny \raggedright{Crespi \& Badcock 2008 Behav Brain Sci.}}
%\end{center}
%\end{frame}

%% clarifying the functional roles of imprinted genes requires characterization of their variation
%% variation across devel. time, tissue type, gender,...
%% variation in postnatal age, aging unclear especially in humans
%% theoretical and experimental mouse studies
%\begin{frame}[label=previous-age-studies]{Variation with age}
%\begin{columns}[t]
%\begin{column}{0.5\textwidth}
%
%\includegraphics[height=0.5\textheight]{figures/from-others/ubeda-2012-fig3a.jpg}
%
%{\tiny Ubeda 2012 Evolution}
%\end{column}
%
%\begin{column}{0.5\textwidth}
%
%\includegraphics[height=0.3\textheight]{figures/from-others/perez-2015-elife-fig4b.png}
%
%{\tiny Perez et al 2015 eLife}
%\end{column}
%\end{columns}
%\end{frame}

\begin{frame}<1>[label=questions]{\textit{Questions} \only<2->{and \textbf{Answers}}}
\begin{enumerate}
\item Imprinted genes in the DLPFC\footnote{dorsolateral prefrontal
cortex}
\begin{itemize}
\item \textit{How many?} \only<2->{\textbf{30 genes in \(\approx\frac{1}{3}\) genome}}
\item \textit{Novelty?} \only<2->{\textbf{8 new imprinted genes}}
\end{itemize}
\item Dependence of allelic bias on:\\
Age, Dx, Gender, Ancestry.\(x\) (genotype)
\begin{itemize}
\item \textit{How to describe dependency structure?} \only<3>{\textbf{Linear
mixed models}}
\item \textit{Genes affected uniformly?} \only<3>{\textbf{No!}}
\item \textit{Most prominent effects?} \only<3>{\textbf{Gene, Age.Gene, Ances1.Gene}}
\end{itemize}
\end{enumerate}
\end{frame}

%\subsection{Imprinted genes in the brain}

\begin{frame}{The read count ratio approach}

\includegraphics[width=1.0\textwidth]{figures/from-others/castel-2015-fig1.png}

{\tiny Castel et al 2015 GenomeBiology}
\end{frame}

% our study aims to characterize that variation in humans using CMC data
\begin{frame}[label=our-study]{Our research study}
\begin{description}
\item[data/project] Common Mind Consortium
\item[participants] \alert{Ifat Keydar}, Eva Xia, Menachem Fromer, Doug Ruderfer, Ravi Sachinanandam, Andrew Chess
\end{description}
\end{frame}

% observational study: explain the observed variation in allelic bias with variation in age, gender, psych. condition
% e.g. gene g1 has stronger maternal bias in individual i1 than in i2
% allelic bias not directly observed; RNA-seq read count ration S statistic; technical noise
\begin{frame}[label=cmc]{Study setup}
% CommonMind: within cohort variation of age, gender, genotype, Dx; correlate
% with RNA-seq-based measure of allelic bias
\includegraphics[height=0.75\textheight]{figures/by-me/commonmind-rna-seq/commonmind-rna-seq.pdf}
\end{frame}

\begin{frame}[t]{\only<1>{Distribution across individuals and genes}\only<2-3>{Called imprinted genes}}
\begin{columns}[t]
\begin{column}{0.65\textwidth}

\includegraphics<1-2>[height=0.8\textheight]{figures/2016-07-19-genome-wide-S/complex-plot-b-1.png}
\includegraphics<3>[height=0.8\textheight]{figures/2016-08-01-ifats-filters/venn-total-finalf-called-1.pdf}
\end{column}

\begin{column}{0.35\textwidth}

\includegraphics<2-3>[height=0.80\textheight]{figures/2016-08-01-ifats-filters/top-ranking-genes-1.pdf}
\end{column}
\end{columns}
\end{frame}

\againframe<2>{questions}

%\subsection{Dependence of allelic bias}

\begin{frame}{Explaining inter-individual variation}
\begin{columns}[t]
\begin{column}{0.4\textwidth}

\includegraphics[width=1.1\columnwidth]{figures/2016-07-19-genome-wide-S/complex-plot-1.png}
\end{column}

\begin{column}{0.6\textwidth}

\includegraphics[width=1.1\columnwidth]{figures/2016-06-26-trellis-display-of-data/S-age-gender-2}
\end{column}
\end{columns}
\end{frame}

\begin{frame}{Multiple interdependent predictors}
\includegraphics[width=0.9\textheight]{figures/2016-06-26-trellis-display-of-data/evar-scatterplot-matrix-1.png}
\end{frame}

%\begin{frame}[t]{Inferring dependence using regression models}
%\begin{columns}[t]
%\begin{column}{0.4\textwidth}
%
%generalized linear models
%
%\begin{eqnarray*}
%\mathrm{E}[y_g] &=& \mu_g = h^{-1}(X\beta_g) \\
%y &=& \mu_g + \varepsilon_{\mu_g}
%\end{eqnarray*}
%\end{column}
%
%\begin{column}{0.6\textwidth}
%
%\visible<2->{inference for gene \(g\) (PEG3)}
%
%%\includegraphics<2>[width=1.1\columnwidth]{figures/2016-08-21-likelihood-surface/explain-rll-wireframe-1.png}
%\includegraphics<2>[width=\columnwidth]{figures/2016-08-21-likelihood-surface/explain-rll-levelplot-B-1.png}
%\end{column}
%\end{columns}
%
%\end{frame}

\begin{frame}[t]{Model selection and inference}
\begin{columns}[t]
\begin{column}{0.5\textwidth}
\begin{center}

model selection

%\tiny{(simple regression, for demonstration only)}

\includegraphics[width=1.1\columnwidth]{figures/2016-08-23-glm-sampling-distributions/PEG3-1.png}
\end{center}
\end{column}

\begin{column}{0.5\textwidth}
\begin{center}

\only<2>{maximum likelihood inference}

%\tiny{(multiple regression)}


\includegraphics<2>[width=\columnwidth]{figures/2016-08-21-likelihood-surface/explain-rll-levelplot-B-1.png}
\end{center}
\end{column}
\end{columns}
\end{frame}

\begin{frame}{Explained variation of read count ratio}

\includegraphics[width=0.9\textwidth]{figures/2017-02-14-beta-from-mixed-model/tval-varpart-fixed-present-1.pdf}
\end{frame}

\begin{frame}[label=signif-genes]{Genes \(g\) affected by one or more predictors \(p\)
(\(\beta_{pg}\neq 0\))}
%{Test: parametric + non-parametric (permutations)}
\tiny
/home/attila/projects/monoallelic-brain/presentations/2017-02-20-monoall-general/../../results/signif-gene-effects-either-manual-annot.tex
\end{frame}

%\begin{frame}{Different results under similarly well-fitting models}
%
%\includegraphics[height=0.85\textheight]{figures/2016-06-22-extending-anova/unlm-Q-wnlm-Q-compare-1.pdf}
%\end{frame}

%% two predictors interact when the effect of one predictor (X1) depends on the value (a or b) of another (X2) 
%% this is distinct from correlation/non-orthogonality of predictors
%% e.g. if X2=a then X1 has no dependence on Y (beta1=0) but if X2=b then it does % (beta1>0)
%% we found that effect of Age on S depended on both Institution and Gender at various degrees for different genes
%% we didn't model interactions to avoid extra parameters and decrease in power => our conclusions on beta are too simplistic
%\begin{frame}[label=interaction]{Effects appear interdependent}
%%{The effect of one predictor depends on another}
%
%\includegraphics[width=0.9\textwidth]{figures/2016-07-08-conditional-inference/beta-age-cond-wnlm-Q-2-present-1.pdf}
%\end{frame}

\begin{frame}{Multiple levels of variation}
\begin{center}
\includegraphics[height=0.75\textheight]{figures/by-me/commonmind-rna-seq/commonmind-rna-seq.pdf}
\end{center}
\end{frame}

\begin{frame}{Linear Mixed Models: joint models for nested variation}
\begin{columns}[t]
\begin{column}{0.5\textwidth}
\begin{center}
gene-wise marginal approach\\
%a separate model for each gene

\includegraphics[scale=\platefigscale]{figures/by-me/monoall-dependencies-2/obs-simple-general/obs-simple-general.pdf}
\end{center}
\end{column}

\begin{column}{0.5\textwidth}
\begin{center}
joint approach (mixed model)\\
%a single model for all genes

\bigskip\bigskip

\tikz{\node[draw, align=left]{TODO:\\plate diagram for a mixed model}}
%\includegraphics[scale=\platefigscale]{figures/by-me/monoall-dependencies-2/obs-bayesian/obs-bayesian.pdf}
\end{center}
\end{column}
\end{columns}
\end{frame}

\begin{frame}{Preliminary results from mixed models}
\begin{tabular}{ccc}
term        &   Delta.AIC &        p.Chi \\
\hline
\hline
Age         &   1.7975291 &  6.527338e-01 \\
\hline
\textbf{Age.Gene}    & \textbf{-23.4036}    &  \textbf{2.532556e-06} \\
\hline
Ances1      &  -0.2462851 &  1.339356e-01 \\
\textbf{Ances1.Gene} & \textbf{-56.7626802} &  \textbf{4.826726e-13} \\
\hline
Gender.fix  &   1.0152936 &  0.321039630 \\
Gender.ran  &   2.0000000 &  0.999978157 \\
\textbf{Gender.Gene} &  \textbf{-5.7961247} &  \textbf{0.005235841} \\
Dx.fix      &   3.8933639 &  0.948078426 \\
Dx.ran      &   2.0000000 &  0.999972480 \\
Dx.Gene     &   0.4529866 &  0.213576883 \\
\hline
\end{tabular}

\end{frame}

\againframe<3>{questions}

%\subsection{Future work}

%\begin{frame}{Reanalyze dependence?}
%Is it worth? If yes...
%\begin{enumerate}
%\item add more data
%\begin{itemize}
%\visible<2>{\item no solution for biased stats.~approach}
%\end{itemize}
%\item more accurate read counts
%\begin{itemize}
%\visible<2>{\item improve QC: RNA-seq + genotyping}
%\end{itemize}
%\item find a better model
%\begin{itemize}
%\visible<2>{\item implement inference, validate}
%\end{itemize}
%\end{enumerate}
%\end{frame}

\section{Calling somatic variants with precision}

\begin{frame}{The well-known problem: calls depend on calles}
\begin{columns}[t]
\begin{column}{0.5\textwidth}

\includegraphics[width=1.0\columnwidth]{figures/from-others/kim-2013-bmcbioinf-fig1a.png}
\end{column}
\begin{column}{0.5\textwidth}

\includegraphics[width=1.0\columnwidth]{figures/from-others/kim-2013-bmcbioinf-fig1b.png}
\end{column}
\end{columns}
\end{frame}

\begin{frame}{Combining variants with BAYSIC}
{BMC Bioinformatics. 2014; 15: 104.}

\includegraphics[height=0.7\textheight]{figures/from-others/baysic-fig6.png}
\end{frame}

\begin{frame}{VariantMetaCaller (VMC)}
{G\'{e}zsi,..., Antal, BMC Genomics. 2015; 16: 875.}

\includegraphics[width=0.9\textwidth]{figures/from-others/vmc-fig1.png}
\end{frame}

\begin{frame}{Performance and precision}
\begin{columns}[t]
\begin{column}{0.5\textwidth}
performance

\includegraphics[height=0.8\textheight]{figures/from-others/vmc-fig4b-bwa.png}
\end{column}

\begin{column}{0.5\textwidth}
error of precision

\includegraphics[height=0.8\textheight]{figures/from-others/vmc-fig4c-bwa.png}
\end{column}
\end{columns}
\end{frame}

%\begin{frame}{Brain Somatic Mosaicism}
%\begin{columns}[t]
%\begin{column}{0.6\textwidth}
%
%\includegraphics[width=1.0\columnwidth]{figures/from-others/gage-curropsysbio-2016-1.jpg}
%
%{\tiny Paquola, Erwin, Gage 2016}
%\end{column}
%
%\begin{column}{0.4\textwidth}
%challenges with somatic variants:
%\begin{enumerate}
%\item detection\\
%{\footnotesize allelic fraction}
%\item prioritization\\{\footnotesize multiple info}
%\item integration\\{\footnotesize germline vars.}
%\end{enumerate}
%\end{column}
%\end{columns}
%\end{frame}

\end{document}
