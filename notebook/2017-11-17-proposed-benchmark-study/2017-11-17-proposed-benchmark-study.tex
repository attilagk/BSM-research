\documentclass{beamer}
%\documentclass[handout]{beamer}

% language settings
%\usepackage{fontspec, polyglossia}
%\setdefaultlanguage{magyar}

% common packages
\usepackage{amsmath, multimedia, hyperref, color, multirow}
%\usepackage{graphicx}

% TikZ
\usepackage{tikz}
%\usetikzlibrary{arrows.meta, decorations.pathmorphing, decorations.pathreplacing, shapes.geometric,mindmap}
%\usetikzlibrary{shapes.geometric,fadings,bayesnet}

% beamer styles
\mode<presentation>{
\usetheme{Boadilla}
%\usetheme{Antibes}
%\usecolortheme{beaver}
%\usecolortheme{seahorse}
%\usefonttheme{structureitalicserif}
\setbeamercovered{transparent}
}
\setbeamertemplate{blocks}[rounded][shadow=true]
\AtBeginSubsection[]{
  \begin{frame}<beamer>{Contents}
    \tableofcontents[currentsection,currentsubsection]
  \end{frame}
}
%\useoutertheme[]{tree}

% title, etc
\title{Proposed Benchmark Study}
\subtitle{Biological and technical limitations of somatic variant calling}
\author{Brain Somatic Mosaicism project}
\date{}

\begin{document}

\maketitle

\section{Motivation}

\begin{frame}{Terms}
\small
\begin{tabular}{c|l}
variant & a SNV or small indel, germline or somatic \\
allele frequency & of a somatic variant in a tissue sample \\
paired sample & NeuN+ DLPFC nuclei and reference tissue \\
data set & NGS data from a paired sample \\
caller & a classifier yielding candidate variants \\
filter & a tunable classifier yielding likely true variants \\
compound caller & a combination of a \emph{single} caller and multiple filters \\
combined caller & a combination of \emph{multiple} callers and filters \\
\end{tabular}

\vfill

\begin{tabular}{c|c|c}
 & caller & filter \\
\hline
local info across reads & base (including gap) & caller/quality score, depth,... \\
designed for & sensitivity & specificity \\
\end{tabular}
\end{frame}

\begin{frame}{The ``Reference tissue project''}
\begin{itemize}
\item a paired sample shared by labs
\item NGS technology and compound or combined caller specific to each lab
\end{itemize}
\end{frame}

\begin{frame}{Example: the combined caller of U.~Michigan}
\begin{center}
\includegraphics[height=0.8\textheight]{../../bsm-network/presentations/UM-Pipeline-3.pdf}
\end{center}
\end{frame}

\begin{frame}{Amplicon-seq validation}
{validation attempted for 7 of 11 candidate variants}
\begin{center}
\includegraphics[height=0.9\textheight]{../../bsm-network/presentations/BSMN_data_call_101617-summary.pdf}
\end{center}
\end{frame}

\begin{frame}{Concordance of call sets}
\begin{center}
\includegraphics[height=0.7\textheight]{../../bsm-network/presentations/tbae-call-set-comparison.png}
\end{center}
\end{frame}

\begin{frame}{Questions}
\begin{enumerate}
\item experimental design
\begin{itemize}
\item cost vs sequencing depths of NeuN+ and fibroblast samples
\end{itemize}
\item sensitivity and specificity; their dependence on
\begin{itemize}
\item technical variables (quality scores, depth, ...)
\item allelic fraction in neurons and in reference tissue
\end{itemize}
\item optimization of variant calling
\begin{itemize}
\item the most sensitive caller combination and filtering
\item at given specificity (FDR control)
\end{itemize}
\item germline variants
\begin{itemize}
\item dissection of somatic and germline variants
\end{itemize}
\end{enumerate}
\end{frame}

\section{Proposal}

\end{document}


\begin{columns}[t]
\begin{column}{0.5\textwidth}

\end{column}

\begin{column}{0.5\textwidth}

\end{column}
\end{columns}

